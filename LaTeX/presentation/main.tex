\documentclass[mathserif]{beamer}

\usepackage{pgfpages}
%\setbeameroption{show notes on second screen=bottom} % Both

\usepackage{mathtools}
%%% Local Variables:
%%% mode: latex
%%% TeX-master: "main"
%%% End:

\def\R{\mathbb{R}} 
\def\N{\mathbb{N}}
\def\Z{\mathbb{Z}}
\def\K{\mathbb{K}}
\def\C{\mathbb{C}}
\def\Rp{\mathbb{R}_{+}}
\def\F{\mathbb{F}}
\def\I{\mathbb{I}}

\def\iff{~\Leftrightarrow~}

\def\eps{\mathtt{eps}}
% function spaces
\def\CC{\mathcal{C}}
% space of polynomials
\def\P{{\mathcal{P}}}

% integrals, derivatives
\def\d{\mathrm{d}}
\def\dx{\,\mathrm{d}x}
\def\dt{\,\mathrm{d}t} 

\def\rd{\text{rd}}

\def\Df{\mathrm{D}f}

% duality pairing
\def\<{\langle}
\def\>{\rangle}

% Swap the definition of \abs* and \norm*, so that \abs
% and \norm resizes the size of the brackets, and the 
% starred version does not.
\DeclarePairedDelimiter\abs{\lvert}{\rvert}%
\DeclarePairedDelimiter\norm{\lVert}{\rVert}%

\makeatletter
\let\oldabs\abs%
\def\abs{\@ifstar{\oldabs}{\oldabs*}}
%
\let\oldnorm\norm%
\def\norm{\@ifstar{\oldnorm}{\oldnorm*}}
\makeatother


% vertical equals 
\newcommand{\verteq}{\rotatebox{90}{$\,=$}}
% underset with vertical equals
\newcommand{\equalto}[2]{\underset{\scriptstyle\overset{\mkern4mu\verteq}{#2}}{#1}}

% for fractions with bigger elements or nested fractions
\newcommand{\ffrac}[2]{\ensuremath{\frac{\displaystyle #1}{\displaystyle #2}}}

\DeclareMathOperator{\rang}{rang}
\DeclareMathOperator{\cond}{cond}
\DeclareMathOperator{\diag}{diag}
\DeclareMathOperator{\im}{im}
\DeclareMathOperator{\spr}{spr}

\DeclareMathOperator*{\argmax}{arg\,max}
\DeclareMathOperator*{\argmin}{arg\,min}

\newcommand*{\mat}[1]{\bm{#1}}

%%% Local Variables:
%%% mode: latex
%%% TeX-master: "main"
%%% End:


\usepackage{setspace}

\usepackage{tikz}

\title{Using Complex Shifts in Rayleigh Quotient Iteration to Compute Close Eigenvalues}
\author{{Nils Friess}}
\titlegraphic{\vspace{1cm}\includegraphics[height = 0.23 \textheight]{unihei_logo_4c.eps}}
\date{\today}

\begin{document}
{
\setbeamertemplate{footline}{}
\setbeamertemplate{headline}{} 
\begin{frame}
  \titlepage%
  
  \note{Welcome etc. Self contained, but feel free to interrupt and ask questions if anything is unclear.}
\end{frame}
}

\newcommand{\tc}[1]{\raisebox{.5pt}{\textcircled{\raisebox{-.9pt}
      {#1}}}} \addtocounter{framenumber}{-1} {
  \setbeamertemplate{headline}{}
  \begin{frame}
    \frametitle{Contents}
    Using Complex Shifts \hfill\only<4->{\tc{3}}\hspace*{4cm}
    \\[2ex]
    in Rayleigh Quotient Iteration
    \hfill\only<3->{\tc{2}}\hspace*{4cm}
    \\[2ex]
    to Compute Close Eigenvalues\only<2->{\hfill\tc{1}}\hspace*{4cm}


    \note{
      \begin{itemize}
      \item We pick apart the title. From bottom to top: Want to
        compute eigenvalues, so to make sure that everyone is one the
        same page, go over some basic algorithms + define the actual
        type of problem we are interested in (wide variety of
        eigenvalue algorithms)
      \item On step above we have ``in Rayleigh Quotient
        Iteration''. RQI is a simple eigenvalue algorithm that is most
        known for its fast convergence. It converges cubically and we
        will later see what this exactly means.
      \item Finally, we have ``Using complex shifts''. In this last
        part I present a modification
      \end{itemize}
    }
  \end{frame}
}

\begin{frame}
  \begin{center}
    \includegraphics[width=0.6\textwidth]{figures/plots/withoutCRQI}
  \end{center}
\end{frame}

\begin{frame}
  \begin{center}
    \includegraphics[width=0.6\textwidth]{figures/plots/withCRQI}
  \end{center}
\end{frame}

\section{Symmetric Eigenvalue Problem}
\begin{frame}
  Compute one eigenpair $(\lambda, \vec{v})$ of symmetric matrix
  $\mat{A}$, \ie
  \begin{equation*}
    \mat{A}\vec{v} = \lambda \vec{v}\,.
  \end{equation*}
  % \\[4ex]
  % \uncover<2->{Assumptions}
  % \begin{enumerate}[label=(A\arabic*),leftmargin=30pt]
  % \item[(A1)]<2-> $\mat{A}$ is large and sparse.
  % \item[(A2)]<3-> A good approximation of $\vec{v}$ is available.
  % \item[(A3)]<4-> $\lambda$ lies in the interior of the spectrum.
  % \item[(A4)]<5-> The eigenvalues around $\lambda$ are closely
  %   spaced.
  % \end{enumerate}
\end{frame}

\begin{frame}
  \frametitle{Properties}
  \begin{itemize}
  \item The eigenvalues $\lambda_1, \dotsc, \lambda_n$ of
    $\mat{A}$ are real\\[3ex]
  \item The eigenvectors $\vec{v}_1, \dotsc, \vec{v}_n$ of
    $\mat{A}$ are real\\[3ex]
  \item The eigenvectors form an orthogonal basis of $\R^n$
  \end{itemize}
  
  % \pause %
  % \vspace*{2ex}%
\end{frame}

\begin{frame}
  \begin{algorithm*}[H]
    \setstretch{1.35}%
    \DontPrintSemicolon \KwData{Nonzero unit vector $\vec{x}^{(0)}$}
    \For{$k = 1,2,\dotsc$ until convergence}{
      $\vec{x}^{(k)} \gets \mat{A}\vec{x}^{(k-1)}$\; Normalise
      $\vec{x}^{(k)}$\; }
    \caption{Power method}\label{alg:power:method}
  \end{algorithm*}
  \vspace*{4ex}%
  \begin{itemize}
  \item<2-> Converges linearly to $\vec{v}_n$ with rate
    \[
      \rho = \frac{\abs{\lambda_{n-1}}}{\abs{\lambda_n}}\,.
    \]
  \end{itemize}
\end{frame}

\begin{frame}
  Eigenvalues of $\mat{A}^{-1}$ are
  $\displaystyle \frac{1}{\lambda_i}$.

  \pause%
  \vspace*{3ex}%
  Eigenvalues of $\mat{A} - \mu\mat{I}$ are $\lambda_i - \mu$.

  \pause%
  \vspace*{7ex}%
  Eigenvalues of ${(\mat{A} - \mu\mat{I})}^{-1}$ are
  $\displaystyle \frac{1}{\lambda_i - \mu}$.
\end{frame}

\begin{frame}
  \begin{algorithm*}[H]
    \setstretch{1.35}%
    \DontPrintSemicolon \KwData{Nonzero unit vector $\vec{x}^{(0)}$}
    \For{$k = 1,2,\dotsc$ until convergence}{ Solve
      ${(\mat{A} - \sigma \mat{I})}\vec{x}^{(k)} = \vec{x}^{(k-1)}$\;
      Normalise $\vec{x}^{(k)}$\; }
    \caption{Shifted Inverse Iteration}\label{alg:sii}
  \end{algorithm*}
  \vspace*{4ex}%
  % \renewcommand{\labelitemi}{$\bullet$}
  \begin{itemize}
  \item
  \item<2-> Converges linearly to with rate
    \[
      \rho = \frac{\abs{\mu_1 - \sigma}}{\abs{\mu_2 - \sigma}}\,.
    \]
  \end{itemize}
\end{frame}

\section{Classic RQI}
\begin{frame}
  \begin{columns}[t]
    \begin{column}{0.5\textwidth}
      Eigenvector approximation from eigenvalue approximation?
      \\[4ex]
      \uncover<2->{Shifted Inverse Iteration}
    \end{column}%
    \begin{column}{0.5\textwidth}
      Eigenvalue approximation from eigenvector approximation?
      \\[4ex]
      \uncover<3->{%
        Rayleigh Quotient
        \begin{equation*}
          \rq_{\mat{A}}(\vec{x}) = \frac{\vec{x}^\tp  \mat{A} \vec{x}}{\vec{x}^\tp \vec{x}}
        \end{equation*}
      }
    \end{column}
  \end{columns}
  \visible<4->{%
    % \begin{center}
    %   \includegraphics[width=0.2\textwidth]{figures/arrows}
    % \end{center}
    \begin{center}
      \begin{tikzpicture}
        \node[inner sep=0pt] at (0,0)%
        {\includegraphics[width=0.25\textwidth]{figures/arrows}};
        \node at (0,0) {RQI};
      \end{tikzpicture}
    \end{center}
  }
\end{frame}

\begin{frame}
  \begin{equation*}
    \rq_{\mat{A}}(\vec{x}) = \argmin_{\mu \in \C}\norm{\mat{A}\vec{x} - \mu \vec{x}}
  \end{equation*}
  \\[4ex]%
  \begin{equation*}
    \text{Eigenpair $(\lambda, \vec{v})$:\qquad}%
    \abs{\rq_{\mat{A}}(\vec{x}) - \lambda} = \mathcal{O}\left(\norm{\vec{x} - \vec{v}}^2\right)
  \end{equation*}
\end{frame}
\begin{frame}
  \begin{columns}
    \begin{column}{0.75\textwidth}
      \begin{algorithm*}[H]
        \setstretch{1.35}%
        \DontPrintSemicolon%
        \KwData{Nonzero unit vector $\vec{x}^{(0)}$}
        $\mu^{(0)} \gets {\left(\vec{x}^{(0)}\right)}^\tp \mat{A}
        \vec{x}^{(0)}$ \; \For{$k = 1,2, \dotsc$ until convergence}{
          Solve
          ${\left( \mat{A} - \mu^{(k)} \mat{I}\right)} \vec{y}^{(k)} =
          \vec{x}^{(k-1)}$ for $\vec{y}^{(k)}$\;
          $\vec{x}^{(k)} \gets \vec{y}^{(k)} / \norm{\vec{y}^{(k)}}$\;
          $\mu^{(k)} \gets {\left(\vec{x}^{(k)}\right)}^\tp \mat{A}
          \vec{x}^{(k)}$ \; }
        \caption{Rayleigh Quotient Iteration}\label{alg:rqi}
      \end{algorithm*}
    \end{column}
    \begin{column}{0.25\textwidth}
      \begin{equation*}
        \scriptstyle%
        \rq_{\mat{A}}(\vec{x}) = \frac{\vec{x}^\tp  \mat{A} \vec{x}}{\vec{x}^\tp \vec{x}}
      \end{equation*}
    \end{column}
  \end{columns}
\end{frame}

\begin{frame}
  \begin{itemize}
    \setlength\itemsep{2em}
  \item Converges locally cubically
    \begin{itemize}
    \item<2-> $\blacktriangleright$ Correct digits triple
    \end{itemize}
  \item Converges for almost all starting vectors
    \begin{itemize}
    \item<3-> $\blacktriangleright$ Impossible to fail in practice
    \end{itemize}
  \end{itemize}
  \vspace*{6ex} \visible<4->{
    \begin{block}{\vspace*{-3ex}}
      \begin{center}
        Convergence behaviour can be erratic
      \end{center}
    \end{block}
  }
\end{frame}

\begin{frame}
  \begin{columns}
    \begin{column}{0.4\textwidth}
      \begin{center}
        \includegraphics[width=\textwidth]{figures/plots/withCRQI}
      \end{center}
    \end{column}%
    \begin{column}{0.5\textwidth}
      \begin{block}{Observation}
        Convergence depends on initial shift but \textbf{not} initial
        vector
      \end{block}
    \end{column}
  \end{columns}
\end{frame}

\section{Complex RQI}
\begin{frame}
  Idea: Perturb original problem to increase distance between
  eigenvalues.
  \\[4ex]
  \uncover<2->{ Suppose $\vec{u} \approx \vec{v}_k$.
    \begin{equation*}
      \mat{A} \quad \longrightarrow \quad \mat{A} - i \gamma(\mat{I} - \vec{u} \vec{u}^\tp)
    \end{equation*}
  }
\end{frame}

\begin{frame}
  Suppose $\vec{u} \approx \vec{v}_k$. Set
  $\tilde{\mat{A}} = \mat{A} - i \gamma(\mat{I} - \vec{u}
  \vec{u}^\tp)$.
  
  \pause%
  \vspace*{3ex}%
  Vector $\vec{x}$ that is almost parallel to $\vec{u}$:
  \begin{equation*}
    \tilde{\mat{A}} \vec{x} = \mat{A}\vec{x} - i \gamma (\vec{x} - \vec{u}
    \underbrace{\vec{u}^\tp \vec{x}}_{\approx 1}) \approx \mat{A} \vec{x}\,.
  \end{equation*}

  \vspace*{3ex}%
  Vector $\vec{y}$ that is almost perpendicular to $\vec{u}$:
  \begin{equation*}
    \tilde{\mat{A}} \vec{y} = \mat{A}\vec{y} - i \gamma (\vec{y} - \vec{u}
    \underbrace{\vec{u}^\tp \vec{y}}_{\approx 0}) \approx (\mat{A} - i \gamma \mat{I})\vec{y}\,.
  \end{equation*}
\end{frame}

\begin{frame}
  Eigenvectors and eigenvalues of $\tilde{\mat{A}}$:
  \begin{align*}
    \tilde{\mat{A}}\vec{v}_k &\approx \mat{A} \vec{v}_k = \lambda_k \vec{v}_k \\[1em]
    \tilde{\mat{A}}\vec{v}_j &\approx (\mat{A} - i \gamma \mat{I})\vec{v}_j = (\lambda_j - i \gamma) \vec{v}_j
  \end{align*}
  \visible<2->{
    \begin{center}
      \includegraphics[width=0.85\textwidth]{figures/perturbed_spectrum}
    \end{center}
  }
\end{frame}

\begin{frame}
  Write
  $\tilde{\mat{A}} =\tilde{\mat{A}}_{(0)} + \tilde{\mat{A}}_{(1)}$
  with
  \begin{align*}
    \tilde{\mat{A}}_{(0)} &\coloneqq \mat{A} - i \gamma(\mat{I}- \vec{v}_k \vec{v}_k^\tp)\\[1em]
    \tilde{\mat{A}}_{(1)} &\coloneqq i \gamma ( \vec{u} \vec{u}^\tp - \vec{v}_k\vec{v}_k^\tp)
  \end{align*}
  \pause
  \begin{theorem}
    \begin{equation*}
      \lambda_j(\tilde{\mat{A}}) =
      \lambda_j(\mat{A}) + i \gamma\left({\langle \vec{v}_j, \vec{u} \rangle}^2 - 1\right)
      + \mathcal{O}\left(
        \norm{\tilde{\mat{A}}_{(1)}}^2
      \right)\,,
    \end{equation*}
    where
    \begin{equation*}
      \norm{\tilde{\mat{A}}_{(1)}} = \gamma \sqrt{1 - {\langle \vec{u}, \vec{v}_k \rangle}^2}\,.
    \end{equation*}
  \end{theorem}
\end{frame}

% \section{Classic RQI}
% \begin{frame}
%   \frametitle{RQI}
  
% \end{frame}

% \section{Complex RQI}
% \begin{frame}
%   \frametitle{CRQI}
  
% \end{frame}
\end{document}

%%% Local Variables:
%%% mode: latex
%%% TeX-master: t
%%% End:
