\documentclass[mathserif]{beamer}

\usepackage{mathtools}
%%% Local Variables:
%%% mode: latex
%%% TeX-master: "main"
%%% End:

\def\R{\mathbb{R}} 
\def\N{\mathbb{N}}
\def\Z{\mathbb{Z}}
\def\K{\mathbb{K}}
\def\C{\mathbb{C}}
\def\Rp{\mathbb{R}_{+}}
\def\F{\mathbb{F}}
\def\I{\mathbb{I}}

\def\iff{~\Leftrightarrow~}

\def\eps{\mathtt{eps}}
% function spaces
\def\CC{\mathcal{C}}
% space of polynomials
\def\P{{\mathcal{P}}}

% integrals, derivatives
\def\d{\mathrm{d}}
\def\dx{\,\mathrm{d}x}
\def\dt{\,\mathrm{d}t} 

\def\rd{\text{rd}}

\def\Df{\mathrm{D}f}

% duality pairing
\def\<{\langle}
\def\>{\rangle}

% Swap the definition of \abs* and \norm*, so that \abs
% and \norm resizes the size of the brackets, and the 
% starred version does not.
\DeclarePairedDelimiter\abs{\lvert}{\rvert}%
\DeclarePairedDelimiter\norm{\lVert}{\rVert}%

\makeatletter
\let\oldabs\abs%
\def\abs{\@ifstar{\oldabs}{\oldabs*}}
%
\let\oldnorm\norm%
\def\norm{\@ifstar{\oldnorm}{\oldnorm*}}
\makeatother


% vertical equals 
\newcommand{\verteq}{\rotatebox{90}{$\,=$}}
% underset with vertical equals
\newcommand{\equalto}[2]{\underset{\scriptstyle\overset{\mkern4mu\verteq}{#2}}{#1}}

% for fractions with bigger elements or nested fractions
\newcommand{\ffrac}[2]{\ensuremath{\frac{\displaystyle #1}{\displaystyle #2}}}

\DeclareMathOperator{\rang}{rang}
\DeclareMathOperator{\cond}{cond}
\DeclareMathOperator{\diag}{diag}
\DeclareMathOperator{\im}{im}
\DeclareMathOperator{\spr}{spr}

\DeclareMathOperator*{\argmax}{arg\,max}
\DeclareMathOperator*{\argmin}{arg\,min}

\newcommand*{\mat}[1]{\bm{#1}}

%%% Local Variables:
%%% mode: latex
%%% TeX-master: "main"
%%% End:


\title{Using Complex Shifts in Rayleigh Quotient Iteration to Compute Close Eigenvalues}
\author{{Nils Friess}}
\titlegraphic{\vspace{1cm}\includegraphics[height = 0.23 \textheight]{unihei_logo_4c.eps}}

\date{\today}

\begin{document}
{
\setbeamertemplate{footline}{}
\setbeamertemplate{headline}{} 
\begin{frame}
  \titlepage%
\end{frame}
}

\newcommand{\tc}[1]{\raisebox{.5pt}{\textcircled{\raisebox{-.9pt}
      {#1}}}} \addtocounter{framenumber}{-1}
\begin{frame}
  \frametitle{Contents}
  Using Complex Shifts \hfill\only<4->{\tc{3}}\hspace*{4cm}
  \\[2ex]
  in Rayleigh Quotient Iteration \hfill\only<3->{\tc{2}}\hspace*{4cm}
  \\[2ex]
  to Compute Close Eigenvalues\only<2->{\hfill\tc{1}}\hspace*{4cm}
\end{frame}

\section{Symmetric Eigenvalue Problem}
\begin{frame}
  \frametitle{Symmetric Eigenvalue Problem}
  We want to compute one eigenpair $(\lambda, \vec{v})$ of a real
  symmetric $n \times n$ matrix $\mat{A}$
  \begin{equation*}
    \mat{A}\vec{v} = \lambda \vec{v}\,.
  \end{equation*}
  \\[4ex]
  \uncover<2->{Assumptions}
  \begin{enumerate}[label=(A\arabic*),leftmargin=30pt]
  \item[(A1)]<2-> $\mat{A}$ is large and sparse.
  \item[(A2)]<3-> A good approximation of $\vec{v}$ is available.
  \item[(A3)]<4-> $\lambda$ lies in the interior of the spectrum.
  \item[(A4)]<5-> The eigenvalues around $\lambda$ are closely spaced.
  \end{enumerate}
\end{frame}

\begin{frame}
  \frametitle{Power method}
  \begin{algorithm*}[H]
    \DontPrintSemicolon
    \KwData{Nonzero unit vector $\vec{x}^{(0)}$}
    \For{$k = 1,2,\dotsc$ until convergence}{
      $\vec{x}^{(k)} \gets \mat{A}\vec{x}^{(k-1)}$\;
      Normalise $\vec{x}^{(k)}$\;
    }
    \caption{Power method}\label{alg:power:method}
  \end{algorithm*}
  \vspace*{4ex}%
  % \renewcommand{\labelitemi}{$\bullet$}
  \begin{itemize}
  \item<2-> Converges to $\vec{v}_n$ under mild assumptions.
  \item<3-> Converges linearly with rate
    \[
      \rho = \frac{\lambda_{n-1}}{\lambda_n}
    \]
  \end{itemize}
\end{frame}

\begin{frame}
  \frametitle{Shifted Inverse Iteration}
  \begin{algorithm*}[H]
    \DontPrintSemicolon
    \KwData{Nonzero unit vector $\vec{x}^{(0)}$}
    \For{$k = 1,2,\dotsc$ until convergence}{
      Solve ${(\mat{A} - \mu \mat{I})}\vec{x}^{(k)} = \vec{x}^{(k-1)}$\;
      Normalise $\vec{x}^{(k)}$\;
    }
    \caption{Shifted Inverse Iteration}\label{alg:sii}
  \end{algorithm*}
  \vspace*{4ex}%
  % \renewcommand{\labelitemi}{$\bullet$}
  \begin{itemize}
  \item<2-> Converges to $\vec{v}_n$ under mild assumptions.
  \item<3-> Converges linearly with rate
    \[
      \rho = \frac{\lambda_{n-1}}{\lambda_n}
    \]
  \end{itemize}
\end{frame}

\section{Classic RQI}
\begin{frame}
  \frametitle{RQI}
  
\end{frame}

\section{Complex RQI}
\begin{frame}
  \frametitle{CRQI}
  
\end{frame}
\end{document}

%%% Local Variables:
%%% mode: latex
%%% TeX-master: t
%%% End:
