
\documentclass[12pt,a4paper, parskip=full]{scrartcl}

\usepackage[T1]{fontenc}
\usepackage[utf8]{inputenc}
\usepackage[english]{babel}
\usepackage{microtype}
\usepackage[light]{kpfonts}

\usepackage{mathtools}
\usepackage{amsthm}
\usepackage{bm}

\theoremstyle{definition}
\newtheorem{example}{Example} 

\def\R{\mathbb{R}} 
\def\N{\mathbb{N}}
\def\Z{\mathbb{Z}}
\def\K{\mathbb{K}}
\def\C{\mathbb{C}}
\def\Rp{\mathbb{R}_{+}}
\def\F{\mathbb{F}}
\def\I{\mathbb{I}}

\def\iff{~\Leftrightarrow~}

\def\eps{\mathtt{eps}}
% function spaces
\def\CC{\mathcal{C}}
% space of polynomials
\def\P{{\mathcal{P}}}

% integrals, derivatives
\def\d{\mathrm{d}}
\def\dx{\,\mathrm{d}x}
\def\dt{\,\mathrm{d}t} 

\def\rd{\text{rd}}

\def\Df{\mathrm{D}f}

% duality pairing
\def\<{\langle}
\def\>{\rangle}

% Swap the definition of \abs* and \norm*, so that \abs
% and \norm resizes the size of the brackets, and the 
% starred version does not.
\DeclarePairedDelimiter\abs{\lvert}{\rvert}%
\DeclarePairedDelimiter\norm{\lVert}{\rVert}%

\makeatletter
\let\oldabs\abs%
\def\abs{\@ifstar{\oldabs}{\oldabs*}}
%
\let\oldnorm\norm%
\def\norm{\@ifstar{\oldnorm}{\oldnorm*}}
\makeatother


% vertical equals 
\newcommand{\verteq}{\rotatebox{90}{$\,=$}}
% underset with vertical equals
\newcommand{\equalto}[2]{\underset{\scriptstyle\overset{\mkern4mu\verteq}{#2}}{#1}}

% for fractions with bigger elements or nested fractions
\newcommand{\ffrac}[2]{\ensuremath{\frac{\displaystyle #1}{\displaystyle #2}}}

\DeclareMathOperator{\rang}{rang}
\DeclareMathOperator{\cond}{cond}
\DeclareMathOperator{\diag}{diag}
\DeclareMathOperator{\im}{im}
\DeclareMathOperator{\spr}{spr}

\DeclareMathOperator*{\argmax}{arg\,max}
\DeclareMathOperator*{\argmin}{arg\,min}

\newcommand*{\mat}[1]{\bm{#1}}

%%% Local Variables:
%%% mode: latex
%%% TeX-master: "main"
%%% End:


\addtokomafont{section}{\rmfamily\scshape}

\newcommand{\ie}{i.\,e.\ }
\newcommand{\eg}{e.\,g.\ }

\begin{document}

\section*{Introduction}
We consider the symmetric eigenvalue problem, \ie given a symmetric
matrix $A \in \R^{n \times n}$ we want to find $0 \neq x \in \R^{n}$
and $\lambda \in \R$ such that
\begin{equation}
  \label{eq:eigenproblem}
  Ax = \lambda x \,.
\end{equation}
We assume that $A$ is positive definite which implies, since $A$ is
also symmetric, that all eigenvalues of $A$ are positive. In the
following we are especially interested in finding \emph{interior}
eigenvalues of $A$. By this we mean that if the eigenvalues of $A$ are
labeled in decreasing order of magnitude
$\abs{\lambda_1} \ge \abs{\lambda_2} \ge \dotsc \ge \abs{\lambda_n}$ we are
looking for $\lambda_k$ where $1 < k < n$.

\begin{example}[Structural mechanics, discretization of PDEs]
  This example describes an eigenvalue problem that naturally arises
  in the investigation of resonance frequencies\footnote{This example is
  taken from~\cite{}}. We consider the
  oscillations of a string of unit length.

  Consider the Poisson
  equation
  \begin{equation}
    \label{eq:poisson}
    - \Delta u = f \quad \text{in } \Omega, \qquad u = 0 \quad \text{on } \partial \Omega \,,
  \end{equation}
  posed on a domain $\Omega \subset \R^n$. We will now consider the
  one-dimensional case, \ie $n = 1$, with $\Omega =
  (0,1)$. Equation~\eqref{eq:poisson} then simplifies to
  \begin{equation}
    \label{eq:poisson:1D}
    \begin{aligned}
      -u^{\prime \prime}(x) &= f(x) \qquad \text{ for all } x \in (0,1) \,, \\
      u(0) = u(1) &= 0 \,.
    \end{aligned}
  \end{equation}
  Taylor expansion of $u$ around some $x \in (0,1)$ yields the
  \emph{central finite difference quotient}
  \[
    u^{\prime \prime}(x) = \frac{u(x+h) - 2u(x) + u(x-h)}{h^2} +
    \mathcal{O}(h^2).
  \]
  If we only consider~\eqref{eq:poisson:1D} in the discrete mesh
  points $x_i = ih$, $i = 1, \dotsc, m-1$, for $h = m^{-1}$ and
  neglect the $\mathcal{O}(h^2)$ term in the approximation of
  $u^{\prime \prime}$ we can approximate~\eqref{eq:poisson:1D} by
  \begin{equation}
    \label{eq:poisson:approx}
    \frac{1}{h^2}(-u_{i-1} + 2u_i - u_{i+1}) = f(x_i)\,,
  \end{equation}
  where $u_i \coloneqq u(x_i)$ and $u_0 = u_m = 0$. This leads to the
  following system of linear equations
  \begin{equation}
    \label{eq:poisson:sle}
    \mat{A}u \coloneqq \frac{1}{h^2}
    \begin{pmatrix}
      2      & -1     & 0      & \dotsm & \dotsm & 0 \\
      -1     &  2     & -1     & 0      & \dotsm & 0 \\
      0      & -1     & 2      & \ddots & \ddots & \vdots \\
      \vdots & \ddots & \ddots & \ddots & 2      & -1 \\
      0 & \dotsm & \dotsm & 0 & -1 & 2
    \end{pmatrix}
    \begin{pmatrix}
      u_1 \\ u_2 \\ \vdots \\ u_{m-2} \\ u_{m-1}
    \end{pmatrix}
    =
    \begin{pmatrix}
      b_1 \\ b_2 \\ \vdots \\ b_{m-2} \\ b_{m-1}
    \end{pmatrix}
    \coloneqq
    \begin{pmatrix}
      f(x_1) \\ f(x_2) \\ \vdots \\ f(x_{m-2}) \\ f(x_{m-1})
    \end{pmatrix} \,.
  \end{equation}
  Obviously $\mat{A}$ is symmetric. Since it is also strictly
  diagonally dominant with positive entries on the diagonal it follows
  that it is positive definite.
\end{example}

% \subsection*{Inverse iteration}
% We start by revisiting the idea of the inverse iteration algorithm.
% This method allows us to compute arbitrary eigenvalues of $A$. Assume
% for a moment that $A$ is non-singular. Then $\lambda$ is an eigenvalue
% of $A$ if and only if $\lambda^{-1}$ is an eigenvalue of $A^{-1}$,
% since for all $\lambda \in \R \setminus \{0\}$ and $x \in \R^n$
% \begin{equation}
%   \label{eq:eigenvalue:inverse}
%   Ax = \lambda x \iff x = \lambda A^{-1}x \iff \lambda^{-1} x = A^{-1}
%   x \,.
% \end{equation}
% From this fact it follows immediatly that the power method applied to
% $A^{-1}$ should yield a sequence of vectors converging to the
% eigenvalue in modulus if
% \[
%   \abs{\lambda_1} < \abs{\lambda_2} \le \dotsc \le \abs{\lambda_n} \,.
% \]
% To compute arbitrary eigenvalues we introduce a \emph{shift parameter}
% $\sigma \in \R$ and consider the matrix $A - \sigma \I$, where we
% denote by $\I$ the identity matrix. If we consider its eigenvalues we
% observe that they are ``shifted'', \ie
% \[
%   Ax = \lambda x \iff (A - \sigma \I)x = (\lambda - \sigma)x \qquad
%   \text{for all } \sigma \in \R,\ x \in \R^n \,.
% \]
% Together with~\eqref{eq:eigenvalue:inverse} this yields
% \[
%   Ax = \lambda x \iff \frac{1}{\lambda - \sigma} x = (A - \sigma
%   \I)^{-1} x \qquad \lambda \in \R \setminus \{ \sigma \},\ x \in \R^n
%   \,.
% \]
% To compute interior eigenvalues of $A$ we can use this by applying the
% power iteration to the inverse of the matrix $A - \sigma \I$. If
% instead of using a constant shift parameter we instead use an
% approximation gained by the \emph{Rayleigh quotient} we arrive at the
% following algorithm.

% \subsection*{Rayleigh quotient iteration}
% The Rayleigh quotient iteration is based on the inverse iteration
% discussed above.  However, instead of using a constant shift we
% introduce the \emph{Rayleigh quotient}
% \[
%   \mathcal{R}_A(x) \coloneqq \mathcal{R}(x) \coloneqq
%   \frac{x^\intercal A x}{x^\intercal x}
% \]
% and use its value as a shift parameter in every step, where $x$ is the
% current approximated eigenvector. This is justified by the fact that if
% $x$ is an eigenvector if $A$ we have
% \[
%   \mathcal{R}(x) = \frac{x^\intercal A x }{x^\intercal x} = \frac{x^\intercal \lambda x}{x^\intercal x} = \lambda \,,
% \]
% and even if $x$ is not an eigenvector it can be shown that $\mathcal{R}(x)$ is a unique solution to the least squares problem $\mathcal{R}(x) = \argmin_{\mu \in \R} \norm{Ax - \mu x}^2$.

\end{document}
