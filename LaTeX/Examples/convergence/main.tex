\documentclass[12pt,a4paper, parskip=full]{scrartcl}

\usepackage[T1]{fontenc}
\usepackage[utf8]{inputenc}
\usepackage[english]{babel}
\usepackage{microtype}
\usepackage[light]{kpfonts}

\usepackage{mathtools}
\usepackage{amsthm}
\usepackage{bm}
\usepackage{float}
\usepackage{booktabs}
 
\def\R{\mathbb{R}} 
\def\N{\mathbb{N}}
\def\Z{\mathbb{Z}}
\def\K{\mathbb{K}}
\def\C{\mathbb{C}}
\def\Rp{\mathbb{R}_{+}}
\def\F{\mathbb{F}}
\def\I{\mathbb{I}}

\def\iff{~\Leftrightarrow~}

\def\eps{\mathtt{eps}}
% function spaces
\def\CC{\mathcal{C}}
% space of polynomials
\def\P{{\mathcal{P}}}

% integrals, derivatives
\def\d{\mathrm{d}}
\def\dx{\,\mathrm{d}x}
\def\dt{\,\mathrm{d}t} 

\def\rd{\text{rd}}

\def\Df{\mathrm{D}f}

% duality pairing
\def\<{\langle}
\def\>{\rangle}

% Swap the definition of \abs* and \norm*, so that \abs
% and \norm resizes the size of the brackets, and the 
% starred version does not.
\DeclarePairedDelimiter\abs{\lvert}{\rvert}%
\DeclarePairedDelimiter\norm{\lVert}{\rVert}%

\makeatletter
\let\oldabs\abs%
\def\abs{\@ifstar{\oldabs}{\oldabs*}}
%
\let\oldnorm\norm%
\def\norm{\@ifstar{\oldnorm}{\oldnorm*}}
\makeatother


% vertical equals 
\newcommand{\verteq}{\rotatebox{90}{$\,=$}}
% underset with vertical equals
\newcommand{\equalto}[2]{\underset{\scriptstyle\overset{\mkern4mu\verteq}{#2}}{#1}}

% for fractions with bigger elements or nested fractions
\newcommand{\ffrac}[2]{\ensuremath{\frac{\displaystyle #1}{\displaystyle #2}}}

\DeclareMathOperator{\rang}{rang}
\DeclareMathOperator{\cond}{cond}
\DeclareMathOperator{\diag}{diag}
\DeclareMathOperator{\im}{im}
\DeclareMathOperator{\spr}{spr}

\DeclareMathOperator*{\argmax}{arg\,max}
\DeclareMathOperator*{\argmin}{arg\,min}

\newcommand*{\mat}[1]{\bm{#1}}

%%% Local Variables:
%%% mode: latex
%%% TeX-master: "main"
%%% End:


\addtokomafont{section}{\rmfamily\scshape}

\newcommand{\ie}{i.\,e.\ }
\newcommand{\eg}{e.\,g.\ }

\begin{document}
\begin{example}
  Consider the following real symmetric positive definite $4 \times 4$
  matrix
  \[
    \mat{A} = 25 \cdot \diag(-1,2,-1) =
    \begin{pmatrix}
      50 & -25 &   0 &   0  \\
      -25 &  50 & -25 &   0  \\
      0  & -25 &  50 & -25  \\
      0  &   0 & -25 &  50 \\
    \end{pmatrix}
  \]
  that arises, for example, when discretizing a 1D Laplacian by finte
  differences.

  Suppose we want to numerically compute a specific eigenpair of
  $\mat{A}$ using the Rayleigh-Quotient iteration.

  The eigenpairs $(\lambda_i, \vec{v}_i)$, $i = 1, \dotsc, 4$, of
  $\mat{A}$ can be computed analytically. The eigenvalues in
  increasing order of magnitude are
  \begin{equation}
    \label{eq:eigenvals}
    \lambda_1 = 9.549, \quad \lambda_2 = 34.5492, \quad \lambda_3 = 65.4508, \quad \lambda_4 = 90.4508
  \end{equation}
  with associated normalised eigenvectors such that $\norm{v_i}_2 = 1$
  \begin{equation}
    \label{eq:eigvecs}
    \vec{v}_1 = \begin{pmatrix} 0.3717 \\ 0.6015 \\ 0.6015 \\ 0.3717 \end{pmatrix}, \quad
    \vec{v}_2 = \begin{pmatrix} 0.6015 \\ 0.3717 \\ -0.3717 \\ -0.6015 \end{pmatrix}, \quad
    \vec{v}_3 = \begin{pmatrix} 0.6015 \\ -0.3717 \\ -0.3717 \\ 0.6015 \end{pmatrix}, \quad
    \vec{v}_4 = \begin{pmatrix} 0.3717 \\ -0.6015 \\ 0.6015 \\ -0.3717 \end{pmatrix} \,.
  \end{equation}
  Now consider the Rayleigh quotient iteration. Suppose we want to
  compute the second eigenpair but we only have an approximation of
  the eigenvalue but not of the eigenvector. If we chose $\sigma = 35$
  as the initial shift and $x = (0.5 \ 0.5 \ 0.5 \ 0.5)^\tp$ we get
  the following results after the first four iterations:
  \begin{table}[H]
    \centering
    \begin{tabular}{ccc}
      Iteration & Eigenvalue approx.  & Eigenvector approx. \\ \toprule
      0         & $35$                & $(0.5 , 0.5 , 0.5 , 0.5)^\tp$ \\
      1         & $11.6438$           & $(-0.2483 , -0.6621 , -0.6621 , -0.2483)^\tp$ \\
      2         & $9.5524$            & $(0.3764 , 0.5986 , 0.5986 , 0.3764)^\tp$ \\
      3         & $9.5492$            & $( -0.3717 , -0.6015 , -0.6015 , -0.3717)^\tp$ \\
      4         & $9.5492$            & $(0.3717 , 0.6015 , 0.6015 , 0.3717)^\tp$
    \end{tabular}
  \end{table}
  Even tough the initial eigenvalue aproximation was very good the
  algorithm converges to a different eigenvector. This is due to the
  fact, that the initial eigenvalue approximation is very good
  already.
\end{example}

Ideas:
\begin{itemize}
\item Plot unit circle
\item Plot random rqi initvecs and color them by eigenvalue the converged to
\end{itemize}

\end{document}