\thispagestyle{empty}
\noindent\textsc{Abstract}\qquad
Rayleigh Quotient Iteration (RQI) is an algorithm to compute an
approximate eigenvector and corresponding eigenvalue of a real
symmetric matrix. It is well-known that RQI exhibits local cubice
convergence but the behaviour of the method is sometimes erratic.
Even when started with accurate approximations of an eigenvector, the
algorithm might converge to a different eigenpair than the desired one
and this becomes worse when the eigenvalues around the target are
closely spaced. In this thesis we propose a modification of RQI that
ensures convergence to the desired eigenpair provided that the angle
between the initial vector and the target eigenvector is sufficiently
small. Numerical examples suggest that ``sufficiently small'' means
approximately below $45^\circ$---independent of the spacing of the
eigenvalues.

\selectlanguage{ngerman}%

\vfill

\noindent\textsc{Zusammenfassung}\qquad
Die Rayleigh-Quotient-Iteration (RQI) ist ein Algorithmus, um einen
Eigenvektor und zugehörigen Eigenwert einer reellen, symmetrischen
Matrix zu berechnen. Es ist bekannt, dass RQI lokal kubisch
konvergiert, jedoch ist das Verhalten der Methode bisweilen
unberechenbar. Selbst wenn RQI mit guten Approximationen eines
Eigenvektors ausgeführt wird, ist nicht sichergestellt, dass das
berechnete Eigenpaar das gewünschte ist. Dieses Verhalten tritt
besonders dann auf, wenn die Eigenwerte um den gesuchten Eigenwert
einen sehr geringen Abstand zu diesem Eigenwert aufweisen. In dieser
Arbeit schlagen wir eine Abwandlung von RQI vor, die sicherstellt,
dass die Methode das gewünschten Eigenpaar berechnet, vorausgesetzt,
der Winkel zwischen Anfangsvektor und dem Ziel-Eigenvektor ist
hinreichend klein. Numerische Beispiele weisen darauf hin, dass
\glqq{}hinreichend klein\grqq{} bedeutet, dass der Winkel ungefähr
kleiner als $45^\circ$ ist -- unabhängig vom Abstand der Eigenwerte.

\selectlanguage{english}
%%% Local Variables:
%%% mode: latex
%%% TeX-master: "main"
%%% End:
