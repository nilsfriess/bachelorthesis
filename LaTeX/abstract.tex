\thispagestyle{empty}
\paragraph{Abstract\quad}
Rayleigh Quotient Iteration (RQI) is an algorithm to compute an
approximate eigenvector and corresponding eigenvalue of a real
symmetric matrix. It is well-known that RQI exhibits local cubic
convergence but the behaviour of the method is sometimes erratic.
Even when started with accurate approximations of an eigenvalue, the
algorithm might converge to a different eigenpair than the desired one
and this becomes worse when the eigenvalues around the target are
closely spaced. In this thesis we propose a modification of RQI 

\selectlanguage{ngerman}%

\vfill

\paragraph{Zusammenfassung\quad}
Die Rayleigh-Quotient-Iteration (RQI) ist ein Algorithmus um einen
Eigenvektor und zugehörigen Eigenwert einer reellen, symmetrischen
Matrix zu berechnen. Es ist bekannt, dass RQI lokal kubisch
konvergiert, jedoch ist das Verhalten der Methode bisweilen
unberechenbar. Selbst wenn RQI mit guten Approximationen eines
Eigenvektores ausgeführt wird, ist nicht sichergestellt, dass das
berechnete Eigenpaar das gewünschte ist. Dieses Verhalten tritt
besonders dann auf, wenn die Eigenwerte um den gesuchten Eigenwert
einen sehr geringen Abstand zu diesem Eigenwert haben.

\selectlanguage{english}
%%% Local Variables:
%%% mode: latex
%%% TeX-master: "main"
%%% End:
