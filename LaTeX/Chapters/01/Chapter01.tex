\chapter{Introduction}\label{chapter:intro}%
In this thesis we propose a novel shift-and-invert method to compute
an eigenvector and corresponding eigenvalue of a real symmetric
matrix. The method is a modification of the \emph{Rayleigh Quotient
  Iteration} (RQI) designed to overcome some of the drawbacks of
RQI. It is well-known that RQI is often unpredictable in the sense
that even when executed with accurate eigenvector approximations it
does not necessarily yield the correct eigenpair. This problem becomes
worse in cases where the distance between the wanted eigenvalue and
its neighbours is very small. In our new method we perturb the
original problem in such a way that the spacing between the target
eigenvalue and the other eigenvalues is increased and apply RQI to
this perturbed problem. Obviously, this approach alone would lead to
incorrect results. Therefore, the perturbation is successively
decreased until we finally arrive at the original problem and we hope
that by that time the current eigenvalue approximation has become
sufficiently close to the correct result so that RQI succeeds. For
this approach to work, a sufficiently good eigenvector approximation
is needed and our numerical experiments suggest that ``sufficiently
good'' means that the angle between the initial approximation and the
target eigenvector is below $45^\circ$---independent of the spacing of
the eigenvalues. Although the method converges not as fast as RQI,
often merely three or four additional iterations are required compared
to RQI.

The thesis is structured as follows. In this chapter we collect some
basic definitions and results from numerical linear algebra. We also
review some simple iterative methods for eigenvalue problems. Since
our new algorithm is strongly related to RQI, the second chapter is
devoted to a detailed study of the Rayleigh Quotient and RQI. Among
others, we give a proof of the property that makes RQI advantageous
over other methods, viz., the local cubic convergence. In the third
chapter we introduce our novel method that we call \emph{Complex
  Rayleigh Quotient Iteration}. After motivating the method, different
numerical experiments are carried out for various test problems to
better understand its behaviour and to illustrate the good properties
mentioned above.

The aim of this thesis is not to carry out a mathematical analysis of
the method but rather to discuss various numerical examples that
illustrate different behavioural aspects of the algorithm. We leave a
detailed analysis of the method based on our observations for future
work.

% \section{Objectives and structure of this thesis}
\todo{Rewrite}
We have now established all the necessary results from linear algebra
that are crucial to understanding the remaining of the thesis. We have
also introduced the most basic iterative methods for solving the the
eigenvalue problem~\eqref{eq:eigvalproblem} numerically.

In the next Chapter we study the Rayleigh Quotient and the above
mentioned Rayleigh Quotient Iteration. Among others, we give a proof
for the local cubic convergence of RQI and show that RQI converges for
almost all starting vectors.

In the third chapter we introduce a novel shift-and-invert method that
overcomes some of the disadvantages of classic RQI. We provide a wide
variety of numerical examples that are carefully studied and analysed.

%%% Local Variables:
%%% mode: latex
%%% TeX-master: "../../main"
%%% End:
%
\section{The Symmetric Eigenvalue Problem}%
There is a plethora of examples which lead to eigenvalue problems in
almost all of the natural sciences, in engineering but also other
areas such as economics. In many cases the matrix of which the
eigenvalues are sought is real and symmetric. The task of finding
eigenvalues and eigenvectors is then referred to as the
\emph{symmetric eigenvalue problem}. For completeness, we collect some
general facts from linear algebra on eigenvalues and eigenvectors
below.

\begin{definition}
  Let $\mat{A} \in \C^{n \times n}$. A scalar $\lambda \in \C$ is
  called \emph{eigenvalue} of $\mat{A}$ if there exists a nonzero
  vector $\vec{v} \in \C^n$ such that
  \begin{equation}
    \label{eq:eigvalproblem} 
    \mat{A}\vec{v} = \lambda \vec{v}\,.
  \end{equation}
  The vector $\vec{v}$ is called an \emph{eigenvector} of $\mat{A}$
  associated with $\lambda$. The tuple $(\lambda, \vec{v})$ is called
  an \emph{eigenpair}. The set of all eigenvalues of $\mat{A}$ is
  referred to as the \emph{spectrum} and is denoted by
  $\sigma(\mat{A})$. To indicate that eigenvalues belong to a
  particular matrix $\mat{M}$ we sometimes write $\lambda(\mat{M})$.
\end{definition}
Computing eigenpairs is a non-trivial task.
Rewriting~\eqref{eq:eigvalproblem} gives
$\mat{A}\vec{v} - \lambda \vec{v} = \vec{0}$ or
$(\mat{A} - \lambda \mat{I}) \vec{v} = \vec{0}$, where $\mat{I}$ is
the identity matrix. Since $\vec{v}$ cannot be the zero vector, this
equation has a solution if and only if the matrix
$\mat{A} - \lambda \mat{I}$ is singular. Thus, eigenvalues of
$\mat{A}$ are exactly the roots of the \emph{characteristic
  polynomial}
\[
  \chi_{\mat{A}}(t) \coloneqq \det(\mat{A} - t \mat{I}) \,.
\]
This fact, despite being of theoretical importance, cannot be used to
calculate eigenvalues numerically for two reasons. First, the
computation of the coefficients of the polynomial is not
stable~\cite[37]{golub2000eigenvalue}. And even if it was, it is
well-known that even small perturbations in the coefficients of
$\chi_{\mat{A}}(t)$ can lead to devastating errors in the
roots~\cite[190]{trefethen1997}. Thus, other methods are necessary to
solve~\eqref{eq:eigvalproblem} which gave rise to iterative
algorithms. These methods date back to 1846 when Jacobi published a
pioneering paper on a method to compute eigenvalues of symmetric
matrices~\cite{jacobi1846}. Below we present essential facts from
linear algebra preparing us for discussing such iterative methods in
Section~\ref{sec:iterative:algorithms}.

\begin{remark}[Generalisations of eigenvalue problems]
  The problem stated in Equation~\eqref{eq:eigvalproblem} can be
  generalised in multiple ways. Many problems from physics lead to the
  \emph{generalised eigenvalue problem}
  \begin{equation}
    \label{eq:eigvalproblem:general}
    \mat{A} \vec{v} = \lambda \mat{M} \vec{v}\,.
  \end{equation}
  In our case, we have $\mat{M} = \mat{I}$, the identity matrix. Many
  of the numerical algorithms for solving eigenvalue problems of the
  form~\eqref{eq:eigvalproblem} can be modified to
  solve~\eqref{eq:eigvalproblem:general}; often certain assumptions
  have to be posed on $\mat{M}$ such as symmetry and positive
  definiteness.

  Since matrices can be seen as representations of linear operators on
  finite-dimensional vector spaces, we can define eigenvalue problems
  for linear operators on more general spaces that are possibly of
  infinite dimension. The eigenvectors are then usually called
  \emph{eigenfunctions}. Other generalisations include the
  \emph{quadratic eigenvalue problem}
  \begin{equation*}
    (\lambda^2 \mat{A}_2 + \lambda \mat{A}_1) \vec{v} = \mat{A}_0 \vec{v}\,,
  \end{equation*}
  with matrix coefficients
  $\mat{A}_0, \mat{A}_1, \mat{A}_2 \in \C^{n \times n}$ or more
  general \emph{nonlinear eigenproblems}
  \begin{equation*}
    \mat{Q}(\lambda)\vec{v} = \vec{0}\,,
  \end{equation*}
  where $\mat{Q}(\lambda)$ is a nonlinear matrix-valued function.  In
  this thesis, we almost exclusively consider problems of the
  form~\eqref{eq:eigvalproblem}.
  % We will briefly come back to problems
  % of the form~\eqref{eq:eigvalproblem:general} in the last chapter.
\end{remark}

In the following proposition we collect some basic facts on
eigenvalues and eigenvectors. The results are shown under the
assumption that $\mat{A} \in \C^{n \times n}$ is a complex Hermitian
matrix, \ie
$\mat{A} = \mat{A}^\herm \coloneqq \overline{\mat{A}}^\tp$, where the
bar denotes the complex conjugate. If $\mat{A}$ is a real matrix, we
have $\overline{\mat{A}} = \mat{A}$ and thus the following facts hold
in particular for real symmetric matrices.
\begin{proposition}%
  \label{prop:eigval:facts}
  Let $\mat{A} = \mat{A}^\herm \in \C^{n \times n}$ be a Hermitian
  matrix. Denote by $\lambda_1, \lambda_2, \dotsc, \lambda_n$ the
  eigenvalues\footnote{Of course, the eigenvalues need not be
    distinct. But since the eigenvalues of $\mat{A}$ are the roots of
    the $n$-degree polynomial $\chi_{\mat{A}}(t)$, when counting these
    roots with their multiplicity, this polynomial has $n$ roots over
    $\C$. Thus, we can label the eigenvalues from $1$ to $n$.} of
  $\mat{A}$ with associated eigenvectors
  $\vec{v}_1, \dotsc, \vec{v}_n$.
  \begin{enumerate}[label=(\roman*)]
  \item All eigenvalues of $\mat{A}$ are real.
  \item There exists an orthonormal basis of $\C^n$ consisting of
    eigenvectors of $\mat{A}$. If $\mat{A}$ is a real symmetric
    matrix, the eigenvectors form an orthonormal basis of $\R^n$.
    % $\vec{v}_i$ and $\vec{v}_j$ to two distinct eigenvalues
    % $\lambda_i$ and $\lambda_j$ are orthogonal. Hence, after
    % normalisation, we can choose eigenvectors of $\mat{A}$ that form
    % an orthonormal basis of $\R^n$.
  \item If $\mat{A}$ is non-singular the eigenvalues of $\mat{A}^{-1}$
    are given by $\lambda_1^{-1}, \dotsc, \lambda_n^{-1}$ with
    eigenvectors $\vec{v}_1, \dotsc, \vec{v}_n$.
  \item Let $\mu \in \R$ an arbitrary scalar. The eigenvalues of
    $\mat{A} - \mu \mat{I}$ are $\lambda_i - \mu$ with eigenvectors
    $\vec{v}_1, \dotsc, \vec{v}_n$.
  \end{enumerate}
\end{proposition}
\begin{proof}
  Both (i) and (ii) are well-known results from linear algebra and the
  proofs can be found in most standard literature (see, for
  example,~\cite[Theorem 18 and Corollary, p.~314]{hoffmanlinalg}).
  \begin{enumerate}
  \item[(iii)] Suppose $\mat{A}$ is invertible and let
    $(\lambda, \vec{v})$ be an eigenpair of $\mat{A}$ (note that since
    $\mat{A}$ is non-singular we have $\lambda \neq 0$). Then
    \begin{gather*}
      \mat{A} \vec{v} = \lambda \vec{v} \quad \Leftrightarrow \quad
      \mat{A}^{-1} \mat{A} \vec{v} = \lambda \mat{A}^{-1} \vec{v}
      \quad \Leftrightarrow \quad \lambda^{-1} \vec{v} = \mat{A}^{-1}
      \vec{v} \,,
    \end{gather*}
    hence $(\lambda^{-1}, \vec{v})$ is an eigenpair of $\mat{A}^{-1}$.

  \item[(iv)] For $\mu \in \R$ and $(\lambda, \vec{v})$ an eigenpair
    we have
    \begin{gather*}
      \mat{A}\vec{v} = \lambda \vec{v} \quad \Leftrightarrow \quad
      \mat{A}\vec{v} - \mu \vec{v} = \lambda \vec{v} - \mu \vec{v}
      \quad \Leftrightarrow \quad (\mat{A} - \mu \mat{I}) \vec{v} =
      (\lambda - \mu) \vec{v} \,,
    \end{gather*}
    hence $(\lambda - \mu, \vec{v})$ is an eigenpair of
    $\mat{A} - \mu \mat{I}$.
  \end{enumerate}
\end{proof}

In the following, we restrict our attention to the \emph{symmetric
  eigenvalue problem}, \ie we want to find solutions of
Equation~\eqref{eq:eigvalproblem} assuming $\mat{A}$ is a real
symmetric matrix. Thus, unless stated otherwise, for the remainder of
the thesis $\mat{A}$ denotes a matrix of this type. The (real)
eigenvalues are denoted by $\lambda_j(\mat{A}) = \lambda_j$ with
corresponding (real) eigenvectors $\vec{v}_j$. Since any scalar
multiple of an eigenvector is also an eigenvector, we assume that they
are normalised \wrt the Euclidean norm so that
\begin{equation*}
  \norm{\vec{v}_i} \coloneqq \norm{\vec{v}_i}_2
  \coloneqq \sqrt{\vec{v}_i^\tp \vec{v}_i }
  = 1 \quad \text{ for all } i = 1, \dotsc, n\,.
\end{equation*}
Due to Proposition~\ref{prop:eigval:facts} (ii) we have
\begin{equation*}
  \inp{\vec{v}_i}{\vec{v}_j} = \vec{v}_i^\tp \vec{v}_j = 0 \quad
  \text{ for } i \neq j\,,
\end{equation*}
where $\inp{\cdot}{\cdot}$ denotes the Euclidean inner product on
$\R^n$. Since all eigenvalues are real we can label them in increasing
order of magnitude
\begin{equation*}
  \abs{\lambda_1} \le \abs{\lambda_2} \le \dotsc \le \abs{\lambda_n}
  \,.
\end{equation*}
The eigenvalues $\lambda_1$ and $\lambda_n$ are called \emph{extreme}
eigenvalues. The remaining eigenvalues
$\lambda_2, \dotsc, \lambda_{n-1}$ are called \emph{interior}
eigenvalues.

% In addition to the assumption that the matrix we work with is real
% and symmetric, we are mainly interested in cases in which the target
% eigenvalue and its neighbours are very close. We usually do not have
% any a priori knowledge about their location. Also, the matrices are
% assumed to be large such that the computation of the complete set of
% eigenpairs is too expensive. However, we assume a good approximation
% of the target eigenvector is available. As we will see, traditional
% methods do not perform well when the gap between the wanted
% eigenvalue and adjacent eigenvalues is too small.

% Sometimes we will make the assumption that $\mat{A}$ is positive
% definite, \ie that
% \[
%   x^\tp A x > 0 \quad \text{ for all } x \in \R^n \setminus \{
%   \vec{0} \}.
% \]
% Given an eigenvalue $\lambda$ and an associated normalised
% eigenvector $\vec{v}$ we then have
% \[
%   \lambda = \vec{v}^\tp \lambda \vec{v} = \vec{v}^\tp \mat{A}
%   \vec{v} > 0 \,,
% \]
% \ie eigenvalues of symmetric positive definite matrices are
% positive.

%%% Local Variables:
%%% mode: latex
%%% TeX-master: "../../main"
%%% End:
%
\section{Iterative methods for eigenvalue problems}%
\label{sec:iterative:algorithms}
With the necessary facts from linear algebra at hand we can introduce
some simple iterative methods for computing eigenpairs of symmetric
matrices. We are always interested in how fast these methods produce
good approximations of eigenvectors or eigenvalues (or both). The
following definition provides us with a notion of the speed at which a
sequence converges to its limit.

\begin{definition}[Order of Convergence]
  Let ${(\vec{x}^{(k)})}_{k \in \N}$ be a sequence in $\C^n$ that
  converges to $\vec{z} \in \C^n$.
  \begin{enumerate}[label=(\roman*)]
  \item The sequence is said to converge \emph{linearly} to $\vec{z}$,
    if there exists a constant $0 < \rho < 1$ such that
    \begin{equation*}
      \lim_{k \rightarrow \infty} \frac{
        \norm{\vec{x}^{(k+1)} - \vec{z}}
      }{
        \norm{\vec{x}^{(k)} - \vec{z}}
      } < \rho\,,
    \end{equation*}
    where $\rho$ is called the \emph{rate of convergence}.
  \item The sequence \emph{converges with order $q$ to $\vec{z}$} for
    $q > 1$ if
    \begin{equation*}
      \lim_{k \rightarrow \infty} \frac{
        \norm{\vec{x}^{(k+1)} - \vec{z}} 
      }{
        \norm{\vec{x}^{(k)} - \vec{z}}^q
      } < M\,,
    \end{equation*}
    for some $M > 0$. In particular, convergence with order
    \begin{itemize}
    \item $q = 2$ is called \emph{quadratic convergence},
    \item $q = 3$ is called \emph{cubic convergence}
    \end{itemize}
    etc.
  \end{enumerate}
\end{definition}

In some cases, in particular for sequences that approximate
eigenvectors, the convergence behaviour is best studied in terms of
the \emph{error angle} between $\vec{x}^{(k)}$ and $\vec{z}$.

\begin{definition}[Angle]
  The \emph{angle} between two vectors
  $\vec{x}, \vec{y} \in \C^n \setminus \{ \vec{0} \}$ is defined as
  \begin{equation*}
    \angle(\vec{x}, \vec{y}) = \arccos \frac{
      \abs{\langle \vec{x}, \vec{y} \rangle}
    }{
      \norm{\vec{x}} \norm{\vec{y}}
    }\,.
  \end{equation*}
  Often, the following identities are convenient
  \begin{gather*}
    \sin \angle(\vec{x}, \vec{y}) \coloneqq \sqrt{1 - \cos^2
      \angle(\vec{x}, \vec{y})}\,, \qquad \tan \angle(\vec{x},
    \vec{y}) \coloneqq \frac{ \sin \angle(\vec{x}, \vec{y}) }{ \cos
      \angle(\vec{x}, \vec{y}) }\,.
  \end{gather*}
\end{definition}

To see why also the angle can be used to measure the convergence
speed, suppose $\vec{x}^{(k)}$ converges to the unit vector
$\vec{z}$. Let $\vec{u}^{(k)}$ be the unit vector that lies in the
span of $\vec{x}^{(k)}$ and $\vec{z}$ and is orthogonal to $\vec{z}$
and denote by $\phi^{(k)} = \angle(\vec{x}^{(k)}, \vec{z})$ the error
angle between the current vector iterate and the limit. Now, write the
vector iterate $\vec{x}^{(k)}$ as
\begin{equation*}
  \vec{x}^{(k)} = \vec{z} \cos \phi^{(k)} + \vec{u}^{(k)} \sin \phi^{(k)}\,.
\end{equation*}
We temporarily drop the superscripts and write
$\vec{x} = \vec{x}^{(k)}$, $\vec{u} = \vec{u}^{(k)}$ and
$\phi = \phi^{(k)}$. Then, using the identities
$\sin^2(\phi / 2) = \frac{1 - \cos \phi}{2}$,
$\sin^2 \phi + \cos^2 \phi = 1$ and the Pythagorean theorem we obtain
\begin{align*}
  \norm{\vec{x} - \vec{z}}^2 &= \norm{\vec{z} \cos \phi + \vec{u} \sin \phi - \vec{z}}^2 \\
                             &= \norm{\vec{z}(\cos \phi - 1)}^2 + \norm{\vec{u} \sin \phi}^2 \\
                             &= {(\cos \phi - 1)}^2 + 1 - \cos^2 \phi \\
                             &= 2(1 - \cos \phi) = 4 \sin^2( \phi / 2)\,.
\end{align*}
Thus, convergence orders \wrt the norm imply the same convergence
orders in terms of the error angles and vice verca. Note that we did
assume convergence of the sequence.

\subsubsection{Power method}
The \emph{power method} is one of the oldest iterative methods for
computing eigenvectors. It is based on generating the sequence
$\vec{x}^{(k)} \coloneqq \mat{A}^k \vec{x}^{(0)}$ where
$\vec{x}^{(0)}$ is a non-zero unit vector. Of course, $\mat{A}^k$ does
not have to be computed explicitly at each step since
\[
  \mat{A}^k \vec{x} = \mat{A}(\mat{A}(\dotsc\mat{A}(\mat{A}
  \vec{x})\dotsc))\,.
\]
To prevent underflow and overflow errors, $\vec{x}^{(k)}$ is
normalised at each step. In Algorithm~\ref{alg:power:method} we
normalise by ensuring that the largest component of the current
approximation is equal to one. Of course, other norms can be used. The
sequence $\vec{x}^{(k)}$ converges to the eigenvector associated with
the eigenvalue $\lambda_n$ under the assumptions that $\lambda_n$ is
dominant (\ie $\abs{\lambda_n}$ is strictly greater than
$\abs{\lambda_{n-1}}$) and that the starting vector $\vec{x}^{(0)}$
has a non-vanishing component in the direction of $\vec{v}_n$. The
advantage of normalising \wrt the maximum norm is that the largest
component of $\abs*{\mat{A}\vec{x}^{(k-1)}}$ converges to the
eigenvalue $\lambda_n$. Regardless of the normalisation chosen, the
method converges linearly with convergence rate
\begin{equation}
  \label{eq:convergence:power}
  \rho = \frac{\abs{\lambda_{n-1}}}{\abs{\lambda_n}}\,.
\end{equation}
Thus, the method can be very slow if the distance between the
eigenvalues $\lambda_n$ and $\lambda_{n-1}$ is very small. For a
detailed convergence proof, see~\cite[86\psq]{saad2011}.

\begin{algorithm}[htpb]
  \DontPrintSemicolon
  \Begin{ Choose nonzero initial vector $\vec{x}^{(0)}$\;
    \For{$k = 1,2,\dotsc$ until convergence}{
      $\displaystyle \vec{x}^{(k)} =
      \frac{1}{\alpha^{(k)}}\mat{A}\vec{x}^{(k-1)}$\; \tcc{$\alpha^{(k)}$ is the component of $\mat{A}\vec{x}^{(k-1)}$ with the maximum modulus}
    }
  }
  \caption{Power method}\label{alg:power:method}
\end{algorithm}

Besides the possible slow convergence rate, the power method will
always converge to an eigenvector associated with the dominant
eigenvalue $\lambda_n$. In many applications, however, one already has
a good approximation of another eigenvalue and wants to compute an
eigenvector it belongs to. The following method allows for such
computations.

\subsubsection{(Shifted) Inverse Iteration}
The \emph{inverse iteration} is the power method applied to
$\mat{A}^{-1}$ (provided that the inverse exists). Due to
Proposition~\ref{prop:eigval:facts} (iii) this will produce a sequence
of vectors $\vec{x}^{(k)}$ converging to the eigenvector associated to
the eigenvalue that is smallest in modulus $\lambda_1$. Combining this
idea with Proposition~\ref{prop:eigval:facts} (iv) yields the
\emph{shifted inverse iteration}. There, the iterates are defined by
\[
  \vec{x}^{(k)} = \beta {\left( \mat{A} - \sigma \mat{I} \right)}^{-1}
  \vec{x}^{(k-1)}\,,
\]
where $\beta$ is responsible for normalising $\vec{x}^{(k)}$. The
smallest eigenvalue in modulus of the shifted matrix
$\mat{A} - \sigma \mat{I}$ is the eigenvalue of $\mat{A}$ that is
closest to $\sigma$. Hence, this method converges to an eigenvector
associated with this eigenvalue. Of course, the inverse need not be
computed explicitly. Instead, before the loop we can compute the LU
decomposition of $\mat{A} - \sigma \mat{I}$ (or any other
factorisation, if applicable) and solve the system
${\left( \mat{A} - \sigma \mat{I} \right)} \vec{x}^{(k)} =
\vec{x}^{(k-1)}$ for $\vec{x}^{(k)}$. At each step then, only one
backward and one forward substitution is required, reducing the
complexity from $\O(n^3)$ to $\O(n^2)$. We summarise the results in
Algorithm~\ref{alg:sii} (there, we normalise \wrt the Euclidean
norm).\footnote{Note that we did not specify the ``until convergence''
  criteria in neither of the algorithms introduced in this section. We
  postpone this discussion until Section~\ref{sec:rqi:rq}.}

\begin{algorithm}[htbp]
  \DontPrintSemicolon%
  \KwData{Nonzero unit vector $\vec{x}^{(0)}$, shift $\sigma \in \R$}
  Compute $\mat{LU}$ decomposition $\mat{A} - \sigma \mat{I} = \mat{LU}$\;
  \For{$k = 1,2, \dotsc$ until convergence}{
    Solve ${\left( \mat{A} - \sigma \mat{I}\right)} \vec{\tilde{x}}^{(k)} = \vec{x}^{(k-1)}$ for $\vec{\tilde{x}}^{(k)}$\;
    $\vec{x}^{(k)} \gets \vec{\tilde{x}}^{(k)} / \norm*{\vec{\tilde{x}}^{(k)}}$\;
  } 
  \caption{Shifted inverse iteration}\label{alg:sii}
\end{algorithm}
Since this is essentially the power method (applied to the inverse of
$\mat{A} - \sigma \mat{I}$) this algorithm still converges
linearly. However, if we denote by $\mu_1$ the eigenvalue that is
closest to the shift $\sigma$ and by $\mu_2$ the one that is the next
closest one, the eigenvalue of largest modulus of
${(\mat{A} - \sigma \mat{I})}^{-1}$ is $1 / (\mu_1 - \sigma)$
and~\eqref{eq:convergence:power} suggests that the convergence rate is
\[
  \rho = \frac{\abs{\mu_1 - \sigma}}{\abs{\mu_2 - \sigma}} \,.
\]
Therefore, the method is often used to compute an eigenvector of
$\mat{A}$ if a good approximation of the corresponding eigenvalue is
already available.

Note, however, that a shift which is very close to an eigenvalue
produces a very ill-conditioned linear system and one might expect
inverse iteration to fail in these cases since, in general, it is
impossible to solve ill-conditioned systems accurately. Despite this
seemingly sincere problem, in practice it was observed that the method
produces good approximations even in these cases. According to
Parlett~\cite[84\psq]{Parlett1998}, it was Wilkinson who elucidated
why the ill-conditioning is not a problem in most cases. Suppose
$\sigma \approx \lambda$ where $\lambda$ is an eigenvalue of $\mat{A}$
with corresponding eigenvector $\vec{v}$. Wilkinson illustrated that
although $\tilde{\vec{x}}^{(k)}$ may be far from $\vec{v}$, the
normalised solution
$\vec{x}^{(k)} = \tilde{\vec{x}}^{(k)} /
\norm*{\tilde{\vec{x}}^{(k)}}$ will not be far from $\vec{v}$, when
the system is solved backwards-stably, for more details
see~\cite[621--630]{wilkinson},~\cite[68--71]{Parlett1998}
and~\cite{petersWilkinson}. This will become important again later
when we discuss Rayleigh Quotient Iteration. There, the system that is
solved gets increasingly ill-conditioned at each step but for the same
reason as above, in practice this poses no problem.

At each step in the shifted inverse iteration, better approximations
for the target eigenvector are computed. One could try to use these
approximations to compute approximations of the corresponding
eigenvalue and replace occasionally the shift. There are different
techniques to obtain such estimates, \eg the \emph{Wielandt Shifted
  Inverse Iteration} or the \emph{Rayleigh Quotient Iteration}, the
latter of which is rigorously studied in the next chapter. For further
discussion on the variants and developments of these so called
\emph{shift-and-invert} techniques see, e.\,g., the historic
surveys~\cite{ipsenhistory} and~\cite{golub2000eigenvalue} or Sections
2 and 3 of~\cite{tapia2018}.
%%% Local Variables:
%%% mode: latex
%%% TeX-master: "../../main"
%%% End:
%

%%% Local Variables:
%%% mode: latex
%%% TeX-master: "../../main"
%%% End:
