\chapter{Concluding Remarks}
In this thesis we proposed a novel shift-and-invert algorithm to
compute an eigenpair of a real symmetric matrix. We reviewed some
basic eigenvalue algorithms and classic RQI in the first and second
chapter, respectively. We discussed some appealing convergence
properties of RQI but we also saw that it seems to be difficult to
predict its outcome in terms of the initial vector. In the third
chapter, then, we introduced our new method which is designed to
overcome this problem. In order to obtain a more predictable
algorithm, we used the given eigenvector approximation to perturb the
original problem such that the distance between the desired eigenvalue
and its neighbours is increased. We derived bounds for the perturbed
spectrum and saw that the distance is bigger the more accurate the
initial eigenvector guess is. By applying RQI to this perturbed
problem and then successively decreasing the perturbation, we defined
CRQI, our novel method, which behaves much more predictable. This was
illustrated by various numerical examples of which most had very
closely spaced eigenvalues. In particular, we observed that the method
computes the correct result once the initial error angle is below
$45^\circ$.

It goes without saying that the experiments do not replace a detailed
mathematical analysis. Although our novel method is in essence only a
slight modification of classic RQI, it seems unlikely that convergence
results can be transferred easily due to the different behaviour of
the two methods.  Nevertheless, because of the strong relation between
the algorithms, the results that we discussed in
Chapter~\ref{chapter:rqi} could still serve as a starting point for
future work. For instance, it could be studied to which extent the
different results, \eg those characterising convergence
neighbourhoods, are still valid for CRQI.

Although we tried to cover a variety of different problems similar to
those that occur in real-life applications, the test matrices we used
still represent only a limited class of real-world problems. It should
thus be further examined how the method performs in different
applications. As a particular example we suggest the computation of
the eigenvalues of certain self-adjoint differential operators as
discussed in~\cite{marletta2010} and~\cite{marlettascheichl}. The
problems examined by the authors suffer from a problem called
\emph{spectral pollution}: the spectrum of the discretised problem
might contain eigenvalues that are unrelated to spectral properties of
the original problem. These eigenvalues are often so closely spaced
that it is impossible to distinguish them from the eigenvalues of
interest. We refer to these particular papers since the authors
propose a similar approach to ours in that they also add a
complex-valued perturbation to the problem with the goal to increase
the distance between the wanted and unwanted eigenvalues. However,
they perturb the original problem and then solve this perturbed
problem numerically. With our approach we would discretise the
original problem first and only then apply our method.

%%% Local Variables:
%%% mode: latex
%%% TeX-master: "../main"
%%% End:
