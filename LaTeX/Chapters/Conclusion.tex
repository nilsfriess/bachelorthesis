\chapter{Concluding Remarks}
It goes without saying that the experiments we carried out to analyse
the behaviour of CRQI do not replace a detailed mathematical
analysis. Although our novel method is in essence only a slight
modification of classic RQI, it seems unlikely that convergence
results can be transferred easily due to the different behaviour of
the two methods.  Nevertheless, because of the strong relation between
the algorithms the results that we discussed in
Chapter~\ref{chapter:rqi} could still serve as a starting point for
future work. For instance, it could be studied to which extent the
different results, \eg those characterising convergence
neighbourhoods, are still valid for CRQI.

Although we tried to cover a variety of different problems similar to
those that occur in real-life applications, the test matrices we used
still represent only a small class of real-world problems. It should
thus be further examined how the method performs in different
applications. As a particular example we suggest the computation of
the eigenvalues of certain self-adjoint differential operators as
discussed in~\cite{marletta2010} and~\cite{marlettascheichl}. The
problems examined by the authors suffer from a problem called
\emph{spectral pollution}: the spectrum of the discretised problem
might contain eigenvalues that are unrelated to spectral properties of
the original problem. These eigenvalues are often so closely spaced
that it is impossible to distinguish them from the eigenvalues of
interest. We refer to these particular papers since the authors
propose a similar approach to ours in that they also add a
complex-valued perturbation to the problem with the goal to increase
the distance between the wanted and unwanted eigenvalues. However,
they perturb the original problem and then solve this perturbed
problem numerically. With our approach we would discretise the
original problem first and only then apply our method.

%%% Local Variables:
%%% mode: latex
%%% TeX-master: "../main"
%%% End:
