\section{Conclusions}%
In this chapter we have seen that RQI possesses some very favourable
properties. It is practically impossible to make RQI fail to converge
to any eigenpair and convergence is locally cubic.  However, we have
also discussed that in some cases it seems to be difficult, if not
impossible, to predict which eigenpair the method will converge to. As
we will see in the next chapter, for problems with closely spaced
eigenvalues even very accurate eigenvector approximations can lead to
the wrong result. While virtually every publication that discusses RQI
mentions its appealing properties, almost all of them do also mention
this erratic behaviour. For instance, the papers~\cite{beattiefox,
  dax2003} and~\cite{pantazisszyld} refer to the ``unpredictability''
of RQI.  In Parlett's book~\cite{Parlett1998} it is posed as an
exercise to find a $3 \times 3$ example that shows this
behaviour. Beattie and Fox~\cite{beattiefox} and Dax~\cite{dax2003}
claim that this might be one of the reasons why RQI is not as widely
used (another reason being its high cost).

The main purpose of our novel method is to alter RQI in such a way
that this problem is overcome. Although this cannot be done without
loosing some of the beneficial properties of RQI, the observations
from the numerical experiments in the next chapter are very promising.
The algorithm does generally require more iterations but is much more
predictable; once the error angle between the initial vector and the
target eigenvector is sufficiently small, it will always yield the
correct result.

%%% Local Variables:
%%% mode: latex
%%% TeX-master: "../../main"
%%% End:
