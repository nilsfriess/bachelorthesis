Using the Rayleigh quotient to replace the shift in inverse iteration
is justified by the following fact.
\begin{lemma}[Minimal residual]
  Let $\vec{x} \in \R^n \setminus \{ \vec{0} \}$ and define the
  function $f : \R \rightarrow \R$ by
  \[
    f(\lambda) := \norm{(\mat{A} - \lambda \mat {I}) \vec{x}}^2 \,,
  \]
  where $\norm{\cdot} = \norm{\cdot}_2$ denotes the Euclidean norm on
  $\R^n$. The function $f$ becomes minimal for
  $\lambda^\ast = \rq_{\mat{A}}(x)$ with value
  \begin{equation}
    \label{eq:minmal_function_value}
    f(\lambda^\ast) = \norm{\mat{A} \vec{x}}^2 -
    {(\lambda^\ast)}^2 \norm{\vec{x}}^2 \,.
  \end{equation}
\end{lemma}
\begin{proof}
  First, note that
  \begin{align*}
    f(\lambda) = \norm{\mat{A}\vec{x} - \lambda \vec{x}}^2 &= {(\mat{A}\vec{x} - \lambda \vec{x})}^\tp {(\mat{A}\vec{x} - \lambda \vec{x})} \\
                                                           &= \left( {(\mat{A} \vec{x})}^\tp - \lambda \vec{x}^\tp \right) ( \mat{A} \vec{x} - \lambda \vec{x}) \\
                                                           &= \norm{\mat{A}\vec{x}}^2 - 2 \lambda \vec{x}^\tp \mat{A}\vec{x} + \lambda^2 \norm{\vec{x}}^2 \,.
  \end{align*}
  Differentiating this function gives
  \[
    f^\prime(\lambda) = -2 \vec{x}^\tp \mat{A} \vec{v} + 2 \lambda
    \norm{\vec{x}}^2
  \]
  and letting $f^\prime(\lambda^\ast) = 0$ we obtain the only root
  \[
    \lambda^\ast = \frac{\vec{x}^\tp \mat{A}
      \vec{x}}{\norm{\vec{x}}^2} = \frac{\vec{x}^\tp \mat{A}
      \vec{x}}{\vec{x}^\tp \vec{x}} = \rq_{\mat{A}}(\vec{x}) \,.
  \]
  Since $f^{\prime \prime}(\lambda) = 2 \norm{\vec{x}}^2 > 0$ this is
  a minimum and inserting it into $f$ yields the value as given
  in~\eqref{eq:minmal_function_value}.
\end{proof}
This Lemma states that for any vector $\vec{x} \in \R^n$ and scalar
$\gamma \in \R$ it holds that
\[
  \norm{(\mat{A} - \gamma \mat{I}) \vec{x}} \ge \norm{(\mat{A} -
    \rq_{\mat{A}}(\vec{x}) \mat{I})\vec{x}} \eqqcolon
  \norm{\vec{r}_{\mat{A}}(\vec{x})} = \norm{\vec{r}(\vec{x})} \,,
\]\todo{check again if squared or not}
where $\vec{r}(\vec{x})$ is called the \emph{residual vector}. This
motivates us to interpret the Rayleigh Quotient as the value that
``acts most like an eigenvalue'' for $\vec{x}$ in the sense of
minimizing $\norm{\mat{A}\vec{x} - \lambda \vec{x}}$. To make this
claim more quantitative, first notice that if $\vec{x}$ is an
eigenvector, then $\rq_{\mat{A}}(\vec{x}) = \lambda$ is the
corresponding eigenvalue. Taylor expansion of $\rq_{\mat{A}}$ around
$\vec{x} = \vec{v}_j$, where $\vec{v}_j$ is the eigenvector associated
with the $j$-th eigenvalue, yields
\begin{equation}
  \label{eq:rq:taylor}
  \rq_{\mat{A}}(\vec{x}) = \rq_{\mat{A}}(\vec{v}_j) + {(\vec{x} - \vec{v}_j)}^\tp \nabla \rq_{\mat{A}}(\vec{v}_j) + \O(\norm{\vec{x} - \vec{v}_j}^2) \,.
\end{equation}
To compute the gradient $\nabla \rq_{\mat{A}}(\vec{x})$ we compute the
partial derivatives \wrt the coordinates $\vec{x}_i$,
$i = 1, \dotsc, n$, of $\vec{x}$ using the quotient rule
\begin{align*}
  \partial_i \rq_{\mat{A}}(\vec{x}) = \partial_i \left(
  \frac{\vec{x}^\tp \mat{A} \vec{x}}{\vec{x}^\tp \vec{x}} \right) &=
                                                                    \frac{\partial_i (\vec{x}^\tp \mat{A} \vec{x}) \vec{x}^\tp \vec{x} -
                                                                    \vec{x}^\tp \mat{A} \vec{x} \, \partial_i \left( \vec{x}^\tp
                                                                    \vec{x} \right)}{{\left( \vec{x}^\tp \vec{x} \right)}^2} \\
                                                                  &= \frac{
                                                                    \partial_i (\vec{x}^\tp \mat{A} \vec{x})}{\vec{x}^\tp \vec{x}} - \frac{\vec{x}^\tp \mat{A} \vec{x}\, \partial_i(\vec{x}^\tp \vec{x})}{{(\vec{x}^\tp \vec{x})}^2}\,,
\end{align*}
where we used the abbreviation
$\partial_i = \frac{\partial}{\partial \vec{x}_i}$. The nominator of
the first fraction can be computed using the product rule
\begin{align*}
  \partial_i \left( \vec{x}^\tp \mat{A} \vec{x} \right) &= \partial_i \left( \vec{x}^\tp  \right) \, \mat{A}\vec{x} + \vec{x}^\tp \partial_i \left( \mat{A} \vec{x} \right) \\
                                                        &= \vec{e}_i^\tp \mat{A}\vec{x} + \vec{x}^\tp \mat{A} \vec{e}_i \\
                                                        &= \vec{e}_i^\tp \mat{A} \vec{x} + \mat{A}^\tp \vec{x} \vec{e}_i = 2 {(\mat{A} \vec{x})}_i \,,
\end{align*}
where we used the symmetry of $\mat{A}$ in the last step. The
derivative in the second part is
$\partial_i \vec{x}^\tp \vec{x} = 2 \vec{x}_i$ and by collecting all
the partial derivatives in the gradient we obtain
\begin{equation}\label{eq:rq:gradient}
  \begin{aligned}
    \nabla \rq_{\mat{A}}(\vec{x}) &= \frac{2 \mat{A} \vec{x}}{\vec{x}^\tp \vec{x}} - \frac{2 \vec{x}^\tp \mat{A} \vec{x} \vec{x}}{{(\vec{x}^\tp \vec{x})}^2} \\
    &= \frac{2}{\vec{x}^\tp \vec{x}} \left( \mat{A} \vec{x} -
      \rq_{\mat{A}}(\vec{x}) \vec{x} \right) \,.
  \end{aligned}
\end{equation}
If $(\lambda, \vec{v})$ is an eigenpair of $\mat{A}$, we have
\[
  \nabla \rq_{\mat{A}}(\vec{v}) = \frac{2}{\vec{v}^\tp \vec{v}} \left(
    \lambda \vec{v} - \lambda \vec{v}\right) = 0
\]
and inserting this into~\eqref{eq:rq:taylor} yields
\[
  \rq_{\mat{A}}(\vec{x}) - \rq_{\mat{A}}(\vec{q}_j) =
  \mathcal{O}(\norm{\vec{x} - \vec{q}_j}^2) \,,
\]
\ie the Rayleigh quotient is a quadratically accurate estimate of an
eigenvalue.
\begin{remark*}
  We have seen, that for an eigenvector $\vec{v}$ the gradient of the
  Rayleigh quotient is zero. Conversely, if for a vector $\vec{x}$ it
  holds $\nabla \rq_{\mat{A}}(\vec{x}) = 0$, then it follows
  from~\eqref{eq:rq:gradient} that
  \[
    0 = \frac{2}{\vec{x}^\tp \vec{x}} ( \mat{A} \vec{x} -
    \rq_{\mat{A}}(\vec{x}) \vec{x}) \,,
  \]
  \ie by multiplying by $\vec{x}^\tp \vec{x} / 2$
  \[
    \mat{A} \vec{x} = \rq_{\mat{A}}(\vec{x}) \vec{x}.
  \]
  Thus $\nabla \rq_{\mat{A}}(\vec{x})$ is zero if, and only if,
  $\vec{x}$ is an eigenvector of $\mat{A}$.
\end{remark*}
\todo{Define Algorithm}
test~\cite{boerm}.