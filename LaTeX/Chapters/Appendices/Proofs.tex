\chapter{Proofs}\label{appendix:proofs}
\begin{proof}[Proof of Lemma~\ref{lem:eigs:atilde0}]
  First consider $j \neq k$. A straightforward calculation yields
  \begin{equation*}
    (\mat{A} - \gamma i(\mat{I}- \vec{v}_k \vec{v}_k^\tp)) \vec{v}_j = \mat{A}\vec{v}_j - \gamma i \vec{v}_j + \gamma i \vec{v}_k \underbrace{\vec{v}_k^\tp  \vec{v}_j}_{ = 0}
    = \lambda_j \vec{v}_j - \gamma i \vec{v}_j = (\lambda_j - \gamma i) \vec{v}_j \,.   
  \end{equation*}
  For $j = k$ we have
  $\vec{v}_k^\tp \vec{v}_j = \vec{v}_k^\tp \vec{v}_k = 1$ and thus
  \begin{equation*}
    (\mat{A} - \gamma i(\mat{I}- \vec{v}_k \vec{v}_k^\tp)) \vec{v}_k = \lambda_k \vec{v}_k - \gamma i \vec{v}_k + \gamma i \vec{v}_k = \lambda_k \vec{v}_k\,.
  \end{equation*}
\end{proof}


% \begin{lemma}[Sherman-Morrison formula in the complex case]
%   Let $\mat{A} \in \C^{n \times n}$ be invertible. Let
%   $\vec{u}, \vec{v} \in \C^n$ be vectors. If
%   $1 + \vec{v}^\herm \mat{A}^{-1}\vec{u} \neq 0$ then
%   $\mat{A} + \vec{u}\vec{v}^\herm$ is invertible with
%   \begin{equation}%
%     \label{eq:smf}
%     {\left( \mat{A} + \vec{u}\vec{v}^\herm \right)}^{-1} = \mat{A}^{-1} +
%     \frac{
%       \mat{A}^{-1} \vec{u} \vec{v}^\herm \mat{A}^{-1}
%     }{
%       1 + \vec{v}^\herm \mat{A}^{-1} \vec{u}
%     }\,.
%   \end{equation}
% \end{lemma}
% \begin{proof}
%   Denote by $\mat{X}$ the left hand side of~\eqref{eq:smf} and by
%   $\mat{Y}$ the right hand side of the equation. We have to verify
%   $\mat{X}\mat{Y} = \mat{I} = \mat{Y}\mat{X}$.
%   \begin{align*}
%     \mat{X}\mat{Y} &= \left(
%                      \mat{A} + \vec{u}\vec{v}^\herm
%                      \right)
%                      \left(
%                      \mat{A}^{-1} - \frac{
%                      \mat{A}^{-1} \vec{u} \vec{v}^\herm \mat{A}^{-1}
%                      }{
%                      1 + \vec{v}^\herm \mat{A}^{-1} \vec{u}
%                      }
%                      \right) \\
%                    &= \mat{A}\mat{A}^{-1} + \vec{u}\vec{v}^\herm\mat{A}^{-1}
%                      - \frac{
%                      \mat{A}\mat{A}^{-1} \vec{u}\vec{v}^\herm\mat{A}^{-1} + \vec{u}\vec{v}^\herm\mat{A}^{-1}\vec{u}\vec{v}^\herm\mat{A}^{-1}
%                      }{
%                      1 + \vec{v}^\herm \mat{A}^{-1} \vec{u}
%                      } \\
%                    &= \mat{I} + \vec{u}\vec{v}^\herm \mat{A}^{-1}
%                      - \frac{
%                      \vec{u}\left( 1 + \vec{v}^\herm \mat{A}^{-1} \vec{u} \right) \vec{v}^\herm \mat{A}^{-1}
%                      }{
%                      1 + \vec{v}^\herm \mat{A}^{-1} \vec{u}
%                      } \\
%                    &= \mat{I} + \vec{u}\vec{v}^\herm \mat{A}^{-1} - \vec{u}\vec{v}^\herm \mat{A}^{-1} = \mat{I}\,.
%   \end{align*}
%   Similarly
%   \begin{align*}
%     \mat{Y}\mat{X} &=\left(
%                      \mat{A}^{-1} - \frac{
%                      \mat{A}^{-1} \vec{u} \vec{v}^\herm \mat{A}^{-1}
%                      }{
%                      1 + \vec{v}^\herm \mat{A}^{-1} \vec{u}
%                      }
%                      \right)
%                      \left(
%                      \mat{A} + \vec{u}\vec{v}^\herm
%                      \right) \\
%                    &= \mat{A}^{-1} \mat{A} + \mat{A}^{-1}\vec{u}\vec{v}^\herm
%                      - \frac{
%                      \mat{A}^{-1}\vec{u}\vec{v}^\herm \mat{A}^{-1} \mat{A} + \mat{A}^{-1} \vec{u} \vec{v}^\herm \mat{A}^{-1} \vec{u} \vec{v}^\herm
%                      }{
%                      1 + \vec{v}^\herm \mat{A}^{-1} \vec{u}
%                      } \\
%                    &= \mat{I} + \mat{A}^{-1} \vec{u}\vec{v}^\herm
%                      -\frac{
%                      \mat{A}^{-1} \vec{u} (1 + \vec{v}^\herm \mat{A}^{-1} \vec{u})\vec{v}^\herm
%                      }{
%                      1 + \vec{v}^\herm \mat{A}^{-1} \vec{u}
%                      }\\
%                    &= \mat{I} + \mat{A}^{-1}\vec{u}\vec{v}^\herm - \mat{A}^{-1}\vec{u}\vec{v}^\herm = \mat{I}\,,                     
%   \end{align*}
%   which concludes the proof.
% \end{proof}

% \begin{proof}[Proof of Lemma~\ref{lem:rq:quadestimate}]
%   This proof is due to Börm~\cite[70]{boerm}.

%   Let $\alpha \in \C$. By using Lemma~\ref{lem:rq:properties} (ii)
%   we obtain
%   \begin{align*}
%     \abs{\rq_{\mat{A}}(\vec{x}) - \lambda}
%     &=
%       \abs{\rq_{\mat{A} - \lambda \mat{I}}(\vec{x})}\\
%     &=
%       \abs{
%       \frac{\langle \vec{x}, (\mat{A} - \lambda \mat{I}) \vec{x} \rangle}{\langle \vec{x}, \vec{x} \rangle}
%       }\\
%     &=
%       \frac{
%       \abs{
%       \langle \vec{x}, (\mat{A} - \lambda \mat{I})(\vec{x} - \alpha \vec{v}) \rangle
%       }
%       }{
%       \norm{x}^2
%       }
%     \\
%     &=
%       \frac{
%       \abs{
%       \langle {(\mat{A} - \lambda \mat{I})}^\ast \vec{x}, \vec{x} - \alpha \vec{v} \rangle
%       }
%       }{
%       \norm{x}^2
%       }
%     \\
%     &= 
%       \frac{
%       \abs{
%       \langle {(\mat{A} - \lambda \mat{I})}^\ast(\vec{x} - \alpha \vec{v}), \vec{x} - \alpha \vec{v} \rangle
%       }
%       }{
%       \norm{x}^2
%       }
%     \\
%     &=
%       \frac{
%       \abs{
%       \langle {\vec{x} - \alpha \vec{v}, (\mat{A} - \lambda \mat{I})} (\vec{x} - \alpha \vec{v}) \rangle
%       }
%       }{
%       \norm{x}^2
%       }
%     \\
%     &\le
%       \frac{
%       \norm{\vec{x} - \alpha \vec{v}} \norm{\mat{A} - \lambda \mat{I}} \norm{\vec{x} - \alpha \vec{v}}
%       }{
%       \norm{\vec{x}}^2
%       }
%     \\
%     &=
%       \norm{\mat{A} - \lambda \mat{I}} {\left(
%       \frac{
%       \norm{\vec{x} - \alpha \vec{v}}
%       }{
%       \norm{\vec{x}}
%       }
%       \right)}^2
%   \end{align*}
% \end{proof}
% \todo{Finish proof (maybe in appendix?)}


%%% Local Variables:
%%% mode: latex
%%% TeX-master: "../../main"
%%% End:

