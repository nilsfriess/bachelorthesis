\chapter{Complex Rayleigh Quotient Iteration}
Recall, that since the eigenvectors $\vec{v}_1, \dotsc, \vec{v}_n$ of
$\mat{A}$ form an orthonormal basis of $\R^n$, we can write every
$\vec{u} \in \R^n$ as
\[
  \vec{u} = \sum_{i=1}^n \alpha_i \vec{v}_i
\]
for certain $\alpha_1, \dotsc, \alpha_n \in \R$. Suppose now, that
$\vec{u}$ is a good approximation for one of the eigenvectors, say for
$\vec{v}_k$, \ie
\[
  \alpha_k \approx 1 \qquad \text{and} \qquad \alpha_j \approx 0 \
  \text{for } j \neq k \,.
\]
Hence, due to the pairwise orthogonality of the eigenvectors
\begin{equation}
  \label{eq:guess_orthogonal}
  \vec{u}^\tp \vec{v}_j \approx
  \begin{cases}
    0 & \text{ if } j \neq k \,, \\
    1 & \text{ if } j = k \,.
  \end{cases}
\end{equation}
As we have seen before, even good approximations of eigenvectors can
lead to convergence to the wrong eigenpair when the gap between the
wanted eigenvalue and eigenvalues nearby is very small.
% used as an initial guess in the classic Rayleigh Quotient Iteration.
We can use~\eqref{eq:guess_orthogonal} to alter the spectrum of
$\mat{A}$ in such a way that the eigenvalue of $\mat{A}$ corresponding
to $\vec{v}_k$ is ``raised'' out of the spectrum, increasing the
distance to adjacent eigenvalues.

% Recall, that all eigenvalues of $\mat{A}$ are real since $\mat{A}$
% is symmetric.
First, observe that given a vector $\vec{x}$, the projection of
$\vec{x}$ onto than span of a unit vector $\vec{u}$ is
\[
  (\vec{u}^\tp \vec{x})\vec{u} = \vec{u}(\vec{u}^\tp \vec{x})=(\vec{u}
  \vec{u}^\tp) \vec{x}\,,
\]
Hence the orthogonal complement is
\[
  \vec{x} - (\vec{u}^\tp \vec{x}) \vec{u} = \vec{x} - (\vec{u}
  \vec{u}^\tp) \vec{x} = (\mat{I} - \vec{u} \vec{u}^\tp) \vec{x} \,.
\]
Now, consider the matrix
$\tilde{\mat{A}} \coloneqq \mat{A} - \gamma i ( \mat{I} - \vec{u}
\vec{u}^\tp)$ where $\gamma > 0$ is an arbitrary real number and $i$
denotes the imaginary unit. If $\vec{u} = \vec{v}_k$ was an
eigenvector of $\mat{A}$ we would have
\begin{align*}
  \tilde{\mat{A}}\vec{v}_j &= \left(  \mat{A} - \gamma i ( \mat{I} - \vec{v}_k
                             \vec{v}_k^\tp) \right) \vec{v}_j \\
                           &= \mat{A} \vec{v}_j - \gamma i \vec{v}_j + \gamma i \vec{v}_k \vec{v}_k^\tp \vec{v}_j \\
                           &= (\lambda_j - \gamma i)\vec{v}_j + \gamma i \delta_{kj} \vec{v}_k \,,
\end{align*}
where $\delta_{kj}$ denotes the Kronecker delta. In other words, if
$j \neq k$, we have
\[
  \tilde{\mat{A}} \vec{v}_j = (\lambda_j - \gamma i)\vec{v}_j
\]
and if $j = k$ we have
\[
  \tilde{\mat{A}} \vec{v}_j = \tilde{\mat{A}} \vec{v}_k = (\lambda_k -
  \gamma i) \vec{v}_k - \gamma i \vec{u} = \lambda_k \vec{v}_k \,.
\]
Therefore, if $\vec{u}$ is not just an approximation of an eigenvector
but we have $\vec{u} = \vec{v}_k$ for some $k$ we can conclude that
$\tilde{\mat{A}}$ has the same set of eigenvectors as $\mat{A}$ with
corresponding eigenvalues $\lambda_j - \gamma i$ if $j \neq k$ and
$\lambda_j$ if $j = k$. In other words, we ``raised'' all eigenvalues
corresponding to eigenvectors that are orthogonal to $\vec{u}$ from
the real line into the complex plane such that their imaginary parts
are equal to $\gamma$.



\begin{lemma}
  Let $\mat{A} \in \R^{n \times n}$ be a real symmetric matrix and let
  $\vec{u} \in \R^n$ an arbitrary unit vector.  We define the matrices
  \begin{equation}
    \label{eq:rayleigh_quotient_complex}
    \mat{B} \coloneqq {\left(\mat{A} - (\sigma - \gamma i) \mat{I} \right)}
  \end{equation}
  and
  \begin{equation}
    \label{eq:rayleigh_quotient_proj}
    \mat{C} \coloneqq {\left( \mat{A} - \sigma \mat{I} + \gamma i( \mat{I} - \vec{u} \vec{u}^\tp) \right)}
  \end{equation}
  where $i$ is the imaginary unit and $\sigma, \gamma > 0$ are
  arbitrary positive real numbers. Then
  \[
    \mathcal{R}_{\mat{A}}(\mat{B}^{-1}\vec{u}) =
    \mathcal{R}_{\mat{A}}(\mat{C}^{-1}\vec{u}) \,,
  \]
  given that the inverses exist.
\end{lemma}
\begin{proof}
  Without loss of generality, we can assume $\sigma = 0$. Otherwise,
  set $\tilde{\mat{A}} = \mat{A} - \sigma \mat{I}$ and use this matrix
  instead of $\mat{A}$ ($\tilde{\mat{A}}$ is obviously still real and
  symmetric).
\end{proof}


%%% Local Variables:
%%% mode: latex
%%% TeX-master: "../../main"
%%% End:
 