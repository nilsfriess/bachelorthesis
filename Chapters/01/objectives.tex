\section{Objectives and structure of this thesis}
\todo{Rewrite}
We have now established all the necessary results from linear algebra
that are crucial to understanding the remaining of the thesis. We have
also introduced the most basic iterative methods for solving the the
eigenvalue problem~\eqref{eq:eigvalproblem} numerically.

In the next Chapter we study the Rayleigh Quotient and the above
mentioned Rayleigh Quotient Iteration. Among others, we give a proof
for the local cubic convergence of RQI and show that RQI converges for
almost all starting vectors.

In the third chapter we introduce a novel shift-and-invert method that
overcomes some of the disadvantages of classic RQI. We provide a wide
variety of numerical examples that are carefully studied and analysed.

%%% Local Variables:
%%% mode: latex
%%% TeX-master: "../../main"
%%% End:
