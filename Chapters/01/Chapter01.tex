\chapter{Introduction}
In this chapter, we introduce the general problem by discussing an
example that naturally arises in different physical applications. We
will then define that basic notion of the Rayleigh quotient and derive
the Rayleigh quotient iteration (RQI) algorithm. Finally, we provide
an overview of the historic developments of RQI and give a summary of
important results concerning the numerical analysis of RQI.
\section{Motivation}

\section{General problem}
We will now formulate the abstract problem that was derived in the
last section. To that end, we will give a few basics facts and
definitions from linear algebra.
\begin{definition}
  Let $\mat{A} \in \R^{n \times n}$ be a real matrix. A scalar
  $\lambda \in \C$ is called \emph{eigenvalue} of $\mat{A}$ if there
  exists a nonzero vector $v \in \C^n$ such that
  \begin{equation}
    \label{eq:eigvalproblem}
    \mat{A}\vec{v} = \lambda \vec{v}\,.
  \end{equation}
  The vector $\vec{v}$ is called an \emph{eigenvector} of $\mat{A}$
  associated with $\lambda$. The tuple $(\lambda, \vec{v})$ is called
  an \emph{eigenpair}. The set of all eigenvalues of $\mat{A}$ is
  called the \emph{spectrum} and is denoted by $\sigma(\mat{A})$.
\end{definition}
In the following proposition we combine some basic facts on
eigenvalues and eigenvectors of symmetric matrices.
\begin{proposition}
  Let $\mat{A} \in \R^{n \times n}$ be a symmetric matrix. Denote by
  $\lambda_1, \lambda_2, \dotsc, \lambda_n$ the eigenvalues of
  $\mat{A}$ with associated eigenvectors
  $\vec{v}_1, \dotsc, \vec{v}_n$.
  \begin{enumerate}[label=(\arabic*)]
  \item All eigenvalues of $\mat{A}$ are real.
  \item Eigenvectors of $\mat{A}$ form a basis of $\R^n$. Eigenvectors
    $\vec{v}_i$ and $\vec{v}_j$ to two distinct eigenvalues
    $\lambda_i$ and $\lambda_j$ are orthogonal. Hence, after
    normalisation, we can choose eigenvectors of $\mat{A}$ that form
    an orthonormal basis of $\R^n$.
  \item If $\mat{A}$ is non-singular the eigenvalues of $\mat{A}^{-1}$
    are given by $\lambda_1^{-1}, \dotsc, \lambda_n^{-1}$ with
    eigenvectors $\vec{v}_1, \dotsc, \vec{v}_n$.
  \item Let $\sigma \in \R$ an arbitrary scalar. Then the eigenvalues
    of $\mat{A} - \sigma \mat{I}$ are $\lambda_i - \sigma$ with
    eigenvectors $\vec{v}_1, \dotsc, \vec{v}_n$.
  \end{enumerate}
\end{proposition}
The eigenvalues $\lambda_1$ and $\lambda_n$ are called \emph{extreme}
eigenvalues. The remaining eigenvalues
($\lambda_2, \dotsc, \lambda_{n-1}$) are called \emph{interior}
eigenvalues.

A well-known fact from linear algebra states that all eigenvalues
$\{\lambda_1, \lambda_2, \dotsc, \lambda_n\}$ of symmetric matrices,
\ie $\mat{A} = \mat{A}^\tp$, are real and that there exist
eigenvectors $\{\vec{v}_1, \vec{v}_2, \dotsc, \vec{v}_n\}$ that form
an orthogonal basis of $\R^n$.

We will often make the assumption that $\mat{A}$ is positive definite,
denoted by $\mat{A} > 0$, which implies that all eigenvalues are
positive. Usually, the eigenvalues will be labeled in ascending order
of magnitude, \ie if $\mat{A} > 0$ holds then
\[
  0 < \lambda_1 \le \lambda_2 \le \dotsc \le \lambda_n \,.
\]

\section{Iterative methods for eigenvalue problems}
In this section we introduce the \emph{power method}, one of the most
simple methods to compute eigenpairs of symmetric matrices. By simply
altering the initial matrix $\mat{A}$ in the power method we arrive at
the \emph{inverse iteration} that allows us to compute eigenvalues
close to a given value. This will then directly lead us to the
\emph{Rayleigh Quotient Iteration} (or \emph{RQI}, for short).
\subsection{Power method}
% Troughout the entire thesis we will follow the notation
% from~\cite{Parlett1998}.
The power method is based on generating the sequence
$\mat{A}^k \vec{v}_0$ where $\vec{v}_0$ is a non-zero unit vector. Of
course, $\mat{A}^k$ does not have to be computed explicitly since
\[
  \mat{A}^k \vec{x} = \mat{A}(\mat{A}(\dotsc\mat{A}(\mat{A}
  \vec{x})\dotsc))\,.
\]
The sequence $\vec{v}_k$ as generated in
Algorithm~\ref{alg:power:method} convergences to the eigenvector
associated with the eigenvalue $\lambda_n$ (under the assumption that
$\lambda_n$ is the unique dominant eigenvalue, cf.~\cite[Theorem
4.2.1]{Parlett1998}). 
\begin{algorithm}
  \DontPrintSemicolon%
  \KwData{Nonzero unit vector $\vec{v}_0$}
  \For{$k = 1,2, \dotsc$ until convergence}{
    $\vec{\tilde{v}}_k \gets \mat{A} \vec{v}_{k-1}$\;
    $\vec{v}_k \gets \vec{\tilde{v}}_k / \norm{\vec{\tilde{v}}_k}$\;
  } 
  \caption{Power method}\label{alg:power:method}
\end{algorithm}

\subsection{Inverse Iteration}
To compute the eigenvector associated to the smallest eigenvalue $\lambda_1$

\section{History}


%%% Local Variables:
%%% mode: latex
%%% TeX-master: "../main"
%%% End:
