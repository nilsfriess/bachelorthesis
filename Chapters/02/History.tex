\section{History and recent developments}
In this section we provide an overview of the historic developments of
Rayleigh Quotient Iteration. We also discuss recent contributions that
are relevant to this thesis.

\subsection{Chronology of Rayleigh Quotient iteration}
We start by giving an overview on different algorithms that strongly
resemble what we now call Rayleigh Quotient iteration. This section
starts with the first mentioning of the Rayleigh quotient and ends
with the paper the first defines Rayleigh quotient iteration as it is
given in Algorithm~\ref{alg:rqi}. Large parts of this overview are
based on~\cite{tapia2018}.\todo{Page}

\paragraph{1894 --- \textsc{Lord Rayleigh}} In the second edition of
his book titled ``The Theory of Sound'' John William Strutt, 3rd Baron
Rayleigh~\cite[110]{rayleigh}, proposed the following iteration for
imporoving an approximate eigenvector $\vec{x}^{(0)}$:
\begin{equation}
  \label{eq:rayleigh:iteration}
  \text{Solve}\quad (\mat{A} - \rq_{\mat{A}}(\vec{x}^{(i)}) \mat{I})\, \vec{x}^{(i+1)} = \vec{e}_1\,,
\end{equation}
where $\vec{e}_1$ denotes the first natural coordinate vector, \ie the
first column of the $n \times n$ identity matrix and $\vec{x}^{(i)}$
and $\vec{x}^{(i+1)}$ denote the current and next iterate,
respectively. To be precise, Lord Rayleigh considered the
\emph{generalised eigenvalue problem}
\begin{equation}
  \label{eq:general:eig}
  \mat{A}\vec{v} = \lambda \mat{M} \vec{v}
\end{equation}
and so in his text, the iteration reads
\begin{equation*}
  \text{Solve}\quad (\mat{A} - \rq_{\mat{A}}(\vec{x}^{(i)}) \mat{M})\, \vec{x}^{(i+1)} = \vec{e}_1\,.
\end{equation*}

\paragraph{1949 --- \textsc{Kohn}}
In a letter to the editor Walter Kohn~\cite{kohn} suggest the
following iteration
\begin{equation*}
  \text{Solve}\quad (\mat{A} - \rq_{\mat{A}}(\vec{x}^{(i)}) \mat{I}) \vec{x}^{(i+1)} = \vec{e}_k\,,
\end{equation*}
where $\vec{e}_k$ is \emph{any} of the natural coordinate
vectors. Without a rigorous proof Kohn argues that
$\rq_{\mat{A}}(\vec{x}^{(i)})$ converges quadratically to an
eigenvalue of $\mat{A}$ (given that $\vec{x}^{(0)}$ is sufficiently
close to an eigenvector of $\mat{A}$). Despite the similarity
to~\eqref{eq:rayleigh:iteration}, Kohn does not mention Lord
Rayleigh's method and it is not known whether or not he was aware of
it.  \todo{Define notion of (linear, quadratic, cubic) convergence}

\paragraph{1951 --- \textsc{Crandall}}
In a text communicated to the Royal society of London\todo{Full name}
Crandall~\cite{crandall} suggests
\begin{equation}
  \label{eq:unnormalised:rqi}
  \text{Solve}\quad (\mat{A} - \rq_{\mat{A}}(\vec{x}^{(i)}) \mat{I}) \vec{x}^{(i+1)} = \vec{x^{(i)}}\,.
\end{equation}
Actually, Crandall also considered the generalised
eigenproblem~\eqref{eq:general:eig} but again for our purposes it is
sufficient to consider the case $\mat{M} = \mat{I}$. Note, that this
algorithm is RQI without the normalisation step. Based on the (wrong)
assumption that the sequence of vectors $\vec{x}^{(k)}$ converges,
Crandall establishes cubic convergence for this sequence. To see why
the assumption is wrong, we assume the contrary, \ie suppose
$\vec{x}^{(i)} \longrightarrow \vec{v}_k$ for some $k = 1, \dotsc, n$.
From~\eqref{eq:unnormalised:rqi} we have
\begin{equation*}
  \quad \mat{A}\vec{v}_k - \rq_{\mat{A}}(\vec{v}_k) \vec{v}_k = \vec{v}_k
  \qquad \Leftrightarrow \qquad
  \mat{A}\vec{v}_k = (1 + \rq_{\mat{A}}(\vec{v}_k)) \vec{v}_k\,,
\end{equation*}
\ie $\vec{v}_k$ is an eigenvector of $\mat{A}$ with corresponding
eigenvalue $1 + \rq_{\mat{A}}(\vec{v}_k)$. Since we know that for any
eigenvector, the value of $\rq_{\mat{A}}(\vec{v}_k)$ is the eigenvalue
it belongs to we have
\begin{equation*}
  \rq_{\mat{A}}(\vec{v}_k) = 1 + \rq_{\mat{A}}(\vec{v}_k)
\end{equation*}
which is a contradiction.

\paragraph{1957 -- 59 --- \textsc{Ostrowski}}
Alexander Ostrowski published a series of six papers titled ``On the
Convergence of the Rayleigh Quotient Iteration for the Computation of
the Characteristic Roots and
Vectors. I-VI''~\cite{ostrowskiI,ostrowskiII,ostrowskiIII,ostrowskiIV,ostrowskiV,ostrowskiVI}. We
mention the titles here, since it is the first mention of the term
\emph{Rayleigh Quotient Iteration}.

In the first paper the author suggests the iteration
\begin{equation*}
  \text{Solve}\quad (\mat{A} - \rq_{\mat{A}}(\vec{x}^{(i)}) \mat{I}) \vec{x}^{(i+1)} = \bm{\eta}\,,\quad \bm{\eta} \neq \vec{0}\,.
\end{equation*}
He rigorously establishes a \emph{quadratic} convergence rate for the
sequence $\{ \rq_{\mat{A}}(\vec{x}^{(i)}) \}$. He then refers to a
paper of Wielandt~\cite{wielandt} and his \emph{fractional} or
\emph{broken iteration} (German: \emph{gebrochene
  Iteration}). Inspired by Wielandt`s method he proposes the following
iteration
\begin{equation}
  \label{eq:rqi:unnormalised:ostrowski}
  \text{Solve}\quad (\mat{A} - \rq_{\mat{A}}(\vec{x}^{(i)}) \mat{I}) \vec{x}^{(i+1)} = \vec{x}^{(i)}\,,
\end{equation}
starting with an arbitrary non-zero vector $\vec{x}^{(0)}$. He then
gives a rigorous proof of the local \emph{cubic} convergence of the
sequence of Rayleigh quotients
$\mu_i \coloneqq \rq_{\mat{A}}(\vec{x}^{(i)})$, \ie
\begin{equation}
  \label{eq:rqi:cubic}
  \frac{
    \mu_{i+1} - \lambda
  }{
    {(\mu_i - \lambda)}^3
  }
  \longrightarrow \gamma \quad \text{ as } i \rightarrow \infty\,,
\end{equation}
where $\lambda$ is an eigenvalue of $\mat{A}$ and $\gamma$ is a
positive constant. Local convergence here means that $\vec{x}^{(0)}$
is assumed to be near the eigenvector corresponding to $\lambda$.

Note that~\eqref{eq:rqi:unnormalised:ostrowski} is the same algorithm
previously proposed by Crandall given in
Equation~\eqref{eq:unnormalised:rqi}. Ostrowski was not aware of
Crandall's method; however,.while the first paper was in press the
following note was added:

\begin{quotation}
  ``Professor G. Forsythe has directed my attention to a paper by
  S. H. Crandall [\,\ldots]. In particular, Professor Crandall
  establishes the \emph{cubic character} of convergence of $\xi_x$ in
  the rule (28), (29). However he does not arrive at our asymptotic
  formula (46), which is the principal result of our
  paper.''~\cite[241]{ostrowskiI}.\footnote{Here, $\xi_x$ denotes the
    $x$-th iterate of the approximate eigenvector, \ie in our notation
    $\vec{x}^{(x)}$. The rule (28), (29) in Ostrowski's paper
    corresponds to our equation~\eqref{eq:rqi:unnormalised:ostrowski}
    and the asymptotic formula (46) he references is given
    in~\eqref{eq:rqi:cubic}.}
\end{quotation}

In the beginning of the second paper~\cite{ostrowskiII} Ostrowski
discusses this in more detail. More importantly, he also points out in
§21 of the text that in order to assure convergence in the vector
iterates (and not just the Rayleigh Quotients) one needs to
\emph{normalise} the vectors. With this small yet important
modification of Crandall's algorithm he fully defined RQI. This was
all shown under the assumption that the matrix $\mat{A}$ is real
symmetric and Ostrowski mentions that all results remain valid in the
complex Hermitian case.

The third paper~\cite{ostrowskiIII} of the series addresses the
non-symmetric (or non-Hermitian) case for which Ostrowski is also able
to define a method that attains local cubic convergence. This method
uses a generalised notion of the Rayleigh quotient and comes at the
expense of solving two linear systems at each step instead of
one.\todo{Read Ostrowski's other papers}

This concludes the overview on the development of RQI from the first
mention of the Rayleigh Quotient by Lord Rayleigh to the first
definition of RQI and the first rigorous proof of what makes it so
appealing, namely the (local) cubic convergence.

\subsection{Further developments and recent contributions}


%%% Local Variables:
%%% mode: latex
%%% TeX-master: "../../main"
%%% End:
